% The Project Gutenberg Etext of Some Christmas Stories by Dickens
% #50 in our series by Charles Dickens
% 
% 
% Copyright laws are changing all over the world, be sure to check
% the copyright laws for your country before posting these files!!
% 
% Please take a look at the important information in this header.
% We encourage you to keep this file on your own disk, keeping an
% electronic path open for the next readers.  Do not remove this.
% 
% 
% **Welcome To The World of Free Plain Vanilla Electronic Texts**
% 
% **Etexts Readable By Both Humans and By Computers, Since 1971**
% 
% *These Etexts Prepared By Hundreds of Volunteers and Donations*
% 
% Information on contacting Project Gutenberg to get Etexts, and
% further information is included below.  We need your donations.
% 
% 
% Some Christmas Stories
% 
% by Charles Dickens
% 
% September, 1998  [Etext #1467]
% 
% 
% The Project Gutenberg Etext of Some Christmas Stories by Dickens
% ******This file should be named cdscs10.txt or cdscs10.zip******
% 
% Corrected EDITIONS of our etexts get a new NUMBER, cdscs11.txt
% VERSIONS based on separate sources get new LETTER, cdscs10a.txt
% 
% 
% This etext was prepared from the 1911 Chapman and Hall Christmas
% Stories (Volume 1) edition by David Price, email ccx074@coventry.ac.uk
% 
% Project Gutenberg Etexts are usually created from multiple editions,
% all of which are in the Public Domain in the United States, unless a
% copyright notice is included.  Therefore, we do NOT keep these books
% in compliance with any particular paper edition, usually otherwise.
% 
% 
% We are now trying to release all our books one month in advance
% of the official release dates, for time for better editing.
% 
% Please note:  neither this list nor its contents are final till
% midnight of the last day of the month of any such announcement.
% The official release date of all Project Gutenberg Etexts is at
% Midnight, Central Time, of the last day of the stated month.  A
% preliminary version may often be posted for suggestion, comment
% and editing by those who wish to do so.  To be sure you have an
% up to date first edition [xxxxx10x.xxx] please check file sizes
% in the first week of the next month.  Since our ftp program has
% a bug in it that scrambles the date [tried to fix and failed] a
% look at the file size will have to do, but we will try to see a
% new copy has at least one byte more or less.
% 
% 
% Information about Project Gutenberg (one page)
% 
% We produce about two million dollars for each hour we work.  The
% fifty hours is one conservative estimate for how long it we take
% to get any etext selected, entered, proofread, edited, copyright
% searched and analyzed, the copyright letters written, etc.  This
% projected audience is one hundred million readers.  If our value
% per text is nominally estimated at one dollar then we produce $2
% million dollars per hour this year as we release thirty-two text
% files per month, or 384 more Etexts in 1998 for a total of 1500+
% If these reach just 10% of the computerized population, then the
% total should reach over 150 billion Etexts given away.
% 
% The Goal of Project Gutenberg is to Give Away One Trillion Etext
% Files by the December 31, 2001.  [10,000 x 100,000,000=Trillion]
% This is ten thousand titles each to one hundred million readers,
% which is only 10% of the present number of computer users.  2001
% should have at least twice as many computer users as that, so it
% will require us reaching less than 5% of the users in 2001.
% 
% 
% We need your donations more than ever!
% 
% 
% All donations should be made to "Project Gutenberg/CMU": and are
% tax deductible to the extent allowable by law.  (CMU = Carnegie-
% Mellon University).
% 
% For these and other matters, please mail to:
% 
% Project Gutenberg
% P. O. Box  2782
% Champaign, IL 61825
% 
% When all other email fails try our Executive Director:
% Michael S. Hart <hart@pobox.com>
% 
% We would prefer to send you this information by email
% (Internet, Bitnet, Compuserve, ATTMAIL or MCImail).
% 
% ******
% If you have an FTP program (or emulator), please
% FTP directly to the Project Gutenberg archives:
% [Mac users, do NOT point and click. . .type]
% 
% ftp uiarchive.cso.uiuc.edu
% login:  anonymous
% password:  your@login
% cd etext/etext90 through /etext96
% or cd etext/articles [get suggest gut for more information]
% dir [to see files]
% get or mget [to get files. . .set bin for zip files]
% GET INDEX?00.GUT
% for a list of books
% and
% GET NEW GUT for general information
% and
% MGET GUT* for newsletters.
% 
% **Information prepared by the Project Gutenberg legal advisor**
% (Three Pages)
% 
% 
% ***START**THE SMALL PRINT!**FOR PUBLIC DOMAIN ETEXTS**START***
% Why is this "Small Print!" statement here?  You know: lawyers.
% They tell us you might sue us if there is something wrong with
% your copy of this etext, even if you got it for free from
% someone other than us, and even if what's wrong is not our
% fault.  So, among other things, this "Small Print!" statement
% disclaims most of our liability to you.  It also tells you how
% you can distribute copies of this etext if you want to.
% 
% *BEFORE!* YOU USE OR READ THIS ETEXT
% By using or reading any part of this PROJECT GUTENBERG-tm
% etext, you indicate that you understand, agree to and accept
% this "Small Print!" statement.  If you do not, you can receive
% a refund of the money (if any) you paid for this etext by
% sending a request within 30 days of receiving it to the person
% you got it from.  If you received this etext on a physical
% medium (such as a disk), you must return it with your request.
% 
% ABOUT PROJECT GUTENBERG-TM ETEXTS
% This PROJECT GUTENBERG-tm etext, like most PROJECT GUTENBERG-
% tm etexts, is a "public domain" work distributed by Professor
% Michael S. Hart through the Project Gutenberg Association at
% Carnegie-Mellon University (the "Project").  Among other
% things, this means that no one owns a United States copyright
% on or for this work, so the Project (and you!) can copy and
% distribute it in the United States without permission and
% without paying copyright royalties.  Special rules, set forth
% below, apply if you wish to copy and distribute this etext
% under the Project's "PROJECT GUTENBERG" trademark.
% 
% To create these etexts, the Project expends considerable
% efforts to identify, transcribe and proofread public domain
% works.  Despite these efforts, the Project's etexts and any
% medium they may be on may contain "Defects".  Among other
% things, Defects may take the form of incomplete, inaccurate or
% corrupt data, transcription errors, a copyright or other
% intellectual property infringement, a defective or damaged
% disk or other etext medium, a computer virus, or computer
% codes that damage or cannot be read by your equipment.
% 
% LIMITED WARRANTY; DISCLAIMER OF DAMAGES
% But for the "Right of Replacement or Refund" described below,
% [1] the Project (and any other party you may receive this
% etext from as a PROJECT GUTENBERG-tm etext) disclaims all
% liability to you for damages, costs and expenses, including
% legal fees, and [2] YOU HAVE NO REMEDIES FOR NEGLIGENCE OR
% UNDER STRICT LIABILITY, OR FOR BREACH OF WARRANTY OR CONTRACT,
% INCLUDING BUT NOT LIMITED TO INDIRECT, CONSEQUENTIAL, PUNITIVE
% OR INCIDENTAL DAMAGES, EVEN IF YOU GIVE NOTICE OF THE
% POSSIBILITY OF SUCH DAMAGES.
% 
% If you discover a Defect in this etext within 90 days of
% receiving it, you can receive a refund of the money (if any)
% you paid for it by sending an explanatory note within that
% time to the person you received it from.  If you received it
% on a physical medium, you must return it with your note, and
% such person may choose to alternatively give you a replacement
% copy.  If you received it electronically, such person may
% choose to alternatively give you a second opportunity to
% receive it electronically.
% 
% THIS ETEXT IS OTHERWISE PROVIDED TO YOU "AS-IS".  NO OTHER
% WARRANTIES OF ANY KIND, EXPRESS OR IMPLIED, ARE MADE TO YOU AS
% TO THE ETEXT OR ANY MEDIUM IT MAY BE ON, INCLUDING BUT NOT
% LIMITED TO WARRANTIES OF MERCHANTABILITY OR FITNESS FOR A
% PARTICULAR PURPOSE.
% 
% Some states do not allow disclaimers of implied warranties or
% the exclusion or limitation of consequential damages, so the
% above disclaimers and exclusions may not apply to you, and you
% may have other legal rights.
% 
% INDEMNITY
% You will indemnify and hold the Project, its directors,
% officers, members and agents harmless from all liability, cost
% and expense, including legal fees, that arise directly or
% indirectly from any of the following that you do or cause:
% [1] distribution of this etext, [2] alteration, modification,
% or addition to the etext, or [3] any Defect.
% 
% DISTRIBUTION UNDER "PROJECT GUTENBERG-tm"
% You may distribute copies of this etext electronically, or by
% disk, book or any other medium if you either delete this
% "Small Print!" and all other references to Project Gutenberg,
% or:
% 
% [1]  Only give exact copies of it.  Among other things, this
%      requires that you do not remove, alter or modify the
%      etext or this "small print!" statement.  You may however,
%      if you wish, distribute this etext in machine readable
%      binary, compressed, mark-up, or proprietary form,
%      including any form resulting from conversion by word pro-
%      cessing or hypertext software, but only so long as
%      *EITHER*:
% 
%      [*]  The etext, when displayed, is clearly readable, and
%           does *not* contain characters other than those
%           intended by the author of the work, although tilde
%           (~), asterisk (*) and underline (_) characters may
%           be used to convey punctuation intended by the
%           author, and additional characters may be used to
%           indicate hypertext links; OR
% 
%      [*]  The etext may be readily converted by the reader at
%           no expense into plain ASCII, EBCDIC or equivalent
%           form by the program that displays the etext (as is
%           the case, for instance, with most word processors);
%           OR
% 
%      [*]  You provide, or agree to also provide on request at
%           no additional cost, fee or expense, a copy of the
%           etext in its original plain ASCII form (or in EBCDIC
%           or other equivalent proprietary form).
% 
% [2]  Honor the etext refund and replacement provisions of this
%      "Small Print!" statement.
% 
% [3]  Pay a trademark license fee to the Project of 20% of the
%      net profits you derive calculated using the method you
%      already use to calculate your applicable taxes.  If you
%      don't derive profits, no royalty is due.  Royalties are
%      payable to "Project Gutenberg Association/Carnegie-Mellon
%      University" within the 60 days following each
%      date you prepare (or were legally required to prepare)
%      your annual (or equivalent periodic) tax return.
% 
% WHAT IF YOU *WANT* TO SEND MONEY EVEN IF YOU DON'T HAVE TO?
% The Project gratefully accepts contributions in money, time,
% scanning machines, OCR software, public domain etexts, royalty
% free copyright licenses, and every other sort of contribution
% you can think of.  Money should be paid to "Project Gutenberg
% Association / Carnegie-Mellon University".
% 
% *END*THE SMALL PRINT! FOR PUBLIC DOMAIN ETEXTS*Ver.04.29.93*END*

\input gutenberg-toc2.tex



% This etext was prepared from the 1911 Chapman and Hall Christmas
% Stories (Volume 1) edition by David Price, email ccx074@coventry.ac.uk



\begin{document}

% Some Short Christmas Stories by Charles Dickens
\gtitle{Some Short Christmas Stories}

\gauthor{Charles Dickens}

% Contents:
% 
% A Christmas Tree
% What Christmas is as we Grow Older
% The Poor Relation's Story
% The Child's Story
% The Schoolboy's Story
% Nobody's Story



\chapter{A Christmas Tree}



I have been looking on, this evening, at a merry company of children
assembled round that pretty German toy, a Christmas Tree.  The tree
was planted in the middle of a great round table, and towered high
above their heads.  It was brilliantly lighted by a multitude of
little tapers; and everywhere sparkled and glittered with bright
objects.  There were rosy-cheeked dolls, hiding behind the green
leaves; and there were real watches (with movable hands, at least,
and an endless capacity of being wound up) dangling from innumerable
twigs; there were French-polished tables, chairs, bedsteads,
wardrobes, eight-day clocks, and various other articles of domestic
furniture (wonderfully made, in tin, at Wolverhampton), perched
among the boughs, as if in preparation for some fairy housekeeping;
there were jolly, broad-faced little men, much more agreeable in
appearance than many real men---and no wonder, for their heads took
off, and showed them to be full of sugar-plums; there were fiddles
and drums; there were tambourines, books, work-boxes, paint-boxes,
sweetmeat-boxes, peep-show boxes, and all kinds of boxes; there were
trinkets for the elder girls, far brighter than any grown-up gold
and jewels; there were baskets and pincushions in all devices; there
were guns, swords, and banners; there were witches standing in
enchanted rings of pasteboard, to tell fortunes; there were
teetotums, humming-tops, needle-cases, pen-wipers, smelling-bottles,
conversation-cards, bouquet-holders; real fruit, made artificially
dazzling with gold leaf; imitation apples, pears, and walnuts,
crammed with surprises; in short, as a pretty child, before me,
delightedly whispered to another pretty child, her bosom friend,
``There was everything, and more.''  This motley collection of odd
objects, clustering on the tree like magic fruit, and flashing back
the bright looks directed towards it from every side---some of the
diamond-eyes admiring it were hardly on a level with the table, and
a few were languishing in timid wonder on the bosoms of pretty
mothers, aunts, and nurses---made a lively realisation of the fancies
of childhood; and set me thinking how all the trees that grow and
all the things that come into existence on the earth, have their
wild adornments at that well-remembered time.

Being now at home again, and alone, the only person in the house
awake, my thoughts are drawn back, by a fascination which I do not
care to resist, to my own childhood.  I begin to consider, what do
we all remember best upon the branches of the Christmas Tree of our
own young Christmas days, by which we climbed to real life.

Straight, in the middle of the room, cramped in the freedom of its
growth by no encircling walls or soon-reached ceiling, a shadowy
tree arises; and, looking up into the dreamy brightness of its top---%
for I observe in this tree the singular property that it appears to
grow downward towards the earth---I look into my youngest Christmas
recollections!

All toys at first, I find.  Up yonder, among the green holly and red
berries, is the Tumbler with his hands in his pockets, who wouldn't
lie down, but whenever he was put upon the floor, persisted in
rolling his fat body about, until he rolled himself still, and
brought those lobster eyes of his to bear upon me---when I affected
to laugh very much, but in my heart of hearts was extremely doubtful
of him.  Close beside him is that infernal snuff-box, out of which
there sprang a demoniacal Counsellor in a black gown, with an
obnoxious head of hair, and a red cloth mouth, wide open, who was
not to be endured on any terms, but could not be put away either;
for he used suddenly, in a highly magnified state, to fly out of
Mammoth Snuff-boxes in dreams, when least expected.  Nor is the frog
with cobbler's wax on his tail, far off; for there was no knowing
where he wouldn't jump; and when he flew over the candle, and came
upon one's hand with that spotted back---red on a green ground---he
was horrible.  The cardboard lady in a blue-silk skirt, who was
stood up against the candlestick to dance, and whom I see on the
same branch, was milder, and was beautiful; but I can't say as much
for the larger cardboard man, who used to be hung against the wall
and pulled by a string; there was a sinister expression in that nose
of his; and when he got his legs round his neck (which he very often
did), he was ghastly, and not a creature to be alone with.

When did that dreadful Mask first look at me?  Who put it on, and
why was I so frightened that the sight of it is an era in my life?
It is not a hideous visage in itself; it is even meant to be droll,
why then were its stolid features so intolerable?  Surely not
because it hid the wearer's face.  An apron would have done as much;
and though I should have preferred even the apron away, it would not
have been absolutely insupportable, like the mask.  Was it the
immovability of the mask?  The doll's face was immovable, but I was
not afraid of \emph{her}.  Perhaps that fixed and set change coming over a
real face, infused into my quickened heart some remote suggestion
and dread of the universal change that is to come on every face, and
make it still?  Nothing reconciled me to it.  No drummers, from whom
proceeded a melancholy chirping on the turning of a handle; no
regiment of soldiers, with a mute band, taken out of a box, and
fitted, one by one, upon a stiff and lazy little set of lazy-tongs;
no old woman, made of wires and a brown-paper composition, cutting
up a pie for two small children; could give me a permanent comfort,
for a long time.  Nor was it any satisfaction to be shown the Mask,
and see that it was made of paper, or to have it locked up and be
assured that no one wore it.  The mere recollection of that fixed
face, the mere knowledge of its existence anywhere, was sufficient
to awake me in the night all perspiration and horror, with, ``O I
know it's coming!  O the mask!''

I never wondered what the dear old donkey with the panniers---there
he is! was made of, then!  His hide was real to the touch, I
recollect.  And the great black horse with the round red spots all
over him---the horse that I could even get upon---I never wondered
what had brought him to that strange condition, or thought that such
a horse was not commonly seen at Newmarket.  The four horses of no
colour, next to him, that went into the waggon of cheeses, and could
be taken out and stabled under the piano, appear to have bits of
fur-tippet for their tails, and other bits for their manes, and to
stand on pegs instead of legs, but it was not so when they were
brought home for a Christmas present.  They were all right, then;
neither was their harness unceremoniously nailed into their chests,
as appears to be the case now.  The tinkling works of the music-%
cart, I \emph{did} find out, to be made of quill tooth-picks and wire; and
I always thought that little tumbler in his shirt sleeves,
perpetually swarming up one side of a wooden frame, and coming down,
head foremost, on the other, rather a weak-minded person---though
good-natured; but the Jacob's Ladder, next him, made of little
squares of red wood, that went flapping and clattering over one
another, each developing a different picture, and the whole
enlivened by small bells, was a mighty marvel and a great delight.

Ah!  The Doll's house!---of which I was not proprietor, but where I
visited.  I don't admire the Houses of Parliament half so much as
that stone-fronted mansion with real glass windows, and door-steps,
and a real balcony---greener than I ever see now, except at watering
places; and even they afford but a poor imitation.  And though it
\emph{did} open all at once, the entire house-front (which was a blow, I
admit, as cancelling the fiction of a staircase), it was but to shut
it up again, and I could believe.  Even open, there were three
distinct rooms in it:  a sitting-room and bed-room, elegantly
furnished, and best of all, a kitchen, with uncommonly soft fire-%
irons, a plentiful assortment of diminutive utensils---oh, the
warming-pan!---and a tin man-cook in profile, who was always going to
fry two fish.  What Barmecide justice have I done to the noble
feasts wherein the set of wooden platters figured, each with its own
peculiar delicacy, as a ham or turkey, glued tight on to it, and
garnished with something green, which I recollect as moss!  Could
all the Temperance Societies of these later days, united, give me
such a tea-drinking as I have had through the means of yonder little
set of blue crockery, which really would hold liquid (it ran out of
the small wooden cask, I recollect, and tasted of matches), and
which made tea, nectar.  And if the two legs of the ineffectual
little sugar-tongs did tumble over one another, and want purpose,
like Punch's hands, what does it matter?  And if I did once shriek
out, as a poisoned child, and strike the fashionable company with
consternation, by reason of having drunk a little teaspoon,
inadvertently dissolved in too hot tea, I was never the worse for
it, except by a powder!

Upon the next branches of the tree, lower down, hard by the green
roller and miniature gardening-tools, how thick the books begin to
hang.  Thin books, in themselves, at first, but many of them, and
with deliciously smooth covers of bright red or green.  What fat
black letters to begin with!  ``A was an archer, and shot at a frog.''
Of course he was.  He was an apple-pie also, and there he is!  He
was a good many things in his time, was A, and so were most of his
friends, except X, who had so little versatility, that I never knew
him to get beyond Xerxes or Xantippe---like Y, who was always
confined to a Yacht or a Yew Tree; and Z condemned for ever to be a
Zebra or a Zany.  But, now, the very tree itself changes, and
becomes a bean-stalk---the marvellous bean-stalk up which Jack
climbed to the Giant's house!  And now, those dreadfully
interesting, double-headed giants, with their clubs over their
shoulders, begin to stride along the boughs in a perfect throng,
dragging knights and ladies home for dinner by the hair of their
heads.  And Jack---how noble, with his sword of sharpness, and his
shoes of swiftness!  Again those old meditations come upon me as I
gaze up at him; and I debate within myself whether there was more
than one Jack (which I am loth to believe possible), or only one
genuine original admirable Jack, who achieved all the recorded
exploits.

Good for Christmas-time is the ruddy colour of the cloak, in which---%
the tree making a forest of itself for her to trip through, with her
basket---Little Red Riding-Hood comes to me one Christmas Eve to give
me information of the cruelty and treachery of that dissembling Wolf
who ate her grandmother, without making any impression on his
appetite, and then ate her, after making that ferocious joke about
his teeth.  She was my first love.  I felt that if I could have
married Little Red Riding-Hood, I should have known perfect bliss.
But, it was not to be; and there was nothing for it but to look out
the Wolf in the Noah's Ark there, and put him late in the procession
on the table, as a monster who was to be degraded.  O the wonderful
Noah's Ark!  It was not found seaworthy when put in a washing-tub,
and the animals were crammed in at the roof, and needed to have
their legs well shaken down before they could be got in, even there---%
and then, ten to one but they began to tumble out at the door,
which was but imperfectly fastened with a wire latch---but what was
\emph{that} against it!  Consider the noble fly, a size or two smaller than
the elephant:  the lady-bird, the butterfly---all triumphs of art!
Consider the goose, whose feet were so small, and whose balance was
so indifferent, that he usually tumbled forward, and knocked down
all the animal creation.  Consider Noah and his family, like idiotic
tobacco-stoppers; and how the leopard stuck to warm little fingers;
and how the tails of the larger animals used gradually to resolve
themselves into frayed bits of string!

Hush!  Again a forest, and somebody up in a tree---not Robin Hood,
not Valentine, not the Yellow Dwarf (I have passed him and all
Mother Bunch's wonders, without mention), but an Eastern King with a
glittering scimitar and turban.  By Allah! two Eastern Kings, for I
see another, looking over his shoulder!  Down upon the grass, at the
tree's foot, lies the full length of a coal-black Giant, stretched
asleep, with his head in a lady's lap; and near them is a glass box,
fastened with four locks of shining steel, in which he keeps the
lady prisoner when he is awake.  I see the four keys at his girdle
now.  The lady makes signs to the two kings in the tree, who softly
descend.  It is the setting-in of the bright Arabian Nights.

Oh, now all common things become uncommon and enchanted to me.  All
lamps are wonderful; all rings are talismans.  Common flower-pots
are full of treasure, with a little earth scattered on the top;
trees are for Ali Baba to hide in; beef-steaks are to throw down
into the Valley of Diamonds, that the precious stones may stick to
them, and be carried by the eagles to their nests, whence the
traders, with loud cries, will scare them.  Tarts are made,
according to the recipe of the Vizier's son of Bussorah, who turned
pastrycook after he was set down in his drawers at the gate of
Damascus; cobblers are all Mustaphas, and in the habit of sewing up
people cut into four pieces, to whom they are taken blind-fold.

Any iron ring let into stone is the entrance to a cave which only
waits for the magician, and the little fire, and the necromancy,
that will make the earth shake.  All the dates imported come from
the same tree as that unlucky date, with whose shell the merchant
knocked out the eye of the genie's invisible son.  All olives are of
the stock of that fresh fruit, concerning which the Commander of the
Faithful overheard the boy conduct the fictitious trial of the
fraudulent olive merchant; all apples are akin to the apple
purchased (with two others) from the Sultan's gardener for three
sequins, and which the tall black slave stole from the child.  All
dogs are associated with the dog, really a transformed man, who
jumped upon the baker's counter, and put his paw on the piece of bad
money.  All rice recalls the rice which the awful lady, who was a
ghoule, could only peck by grains, because of her nightly feasts in
the burial-place.  My very rocking-horse,---there he is, with his
nostrils turned completely inside-out, indicative of Blood!---should
have a peg in his neck, by virtue thereof to fly away with me, as
the wooden horse did with the Prince of Persia, in the sight of all
his father's Court.

Yes, on every object that I recognise among those upper branches of
my Christmas Tree, I see this fairy light!  When I wake in bed, at
daybreak, on the cold, dark, winter mornings, the white snow dimly
beheld, outside, through the frost on the window-pane, I hear
Dinarzade.  ``Sister, sister, if you are yet awake, I pray you finish
the history of the Young King of the Black Islands.''  Scheherazade
replies, ``If my lord the Sultan will suffer me to live another day,
sister, I will not only finish that, but tell you a more wonderful
story yet.''  Then, the gracious Sultan goes out, giving no orders
for the execution, and we all three breathe again.

At this height of my tree I begin to see, cowering among the leaves---%
it may be born of turkey, or of pudding, or mince pie, or of these
many fancies, jumbled with Robinson Crusoe on his desert island,
Philip Quarll among the monkeys, Sandford and Merton with Mr.\ %
Barlow, Mother Bunch, and the Mask---or it may be the result of
indigestion, assisted by imagination and over-doctoring---a
prodigious nightmare.  It is so exceedingly indistinct, that I don't
know why it's frightful---but I know it is.  I can only make out that
it is an immense array of shapeless things, which appear to be
planted on a vast exaggeration of the lazy-tongs that used to bear
the toy soldiers, and to be slowly coming close to my eyes, and
receding to an immeasurable distance.  When it comes closest, it is
worse.  In connection with it I descry remembrances of winter nights
incredibly long; of being sent early to bed, as a punishment for
some small offence, and waking in two hours, with a sensation of
having been asleep two nights; of the laden hopelessness of morning
ever dawning; and the oppression of a weight of remorse.

And now, I see a wonderful row of little lights rise smoothly out of
the ground, before a vast green curtain.  Now, a bell rings---a magic
bell, which still sounds in my ears unlike all other bells---and
music plays, amidst a buzz of voices, and a fragrant smell of
orange-peel and oil.  Anon, the magic bell commands the music to
cease, and the great green curtain rolls itself up majestically, and
The Play begins!  The devoted dog of Montargis avenges the death of
his master, foully murdered in the Forest of Bondy; and a humorous
Peasant with a red nose and a very little hat, whom I take from this
hour forth to my bosom as a friend (I think he was a Waiter or an
Hostler at a village Inn, but many years have passed since he and I
have met), remarks that the sassigassity of that dog is indeed
surprising; and evermore this jocular conceit will live in my
remembrance fresh and unfading, overtopping all possible jokes, unto
the end of time.  Or now, I learn with bitter tears how poor Jane
Shore, dressed all in white, and with her brown hair hanging down,
went starving through the streets; or how George Barnwell killed the
worthiest uncle that ever man had, and was afterwards so sorry for
it that he ought to have been let off.  Comes swift to comfort me,
the Pantomime---stupendous Phenomenon!---when clowns are shot from
loaded mortars into the great chandelier, bright constellation that
it is; when Harlequins, covered all over with scales of pure gold,
twist and sparkle, like amazing fish; when Pantaloon (whom I deem it
no irreverence to compare in my own mind to my grandfather) puts
red-hot pokers in his pocket, and cries ``Here's somebody coming!'' or
taxes the Clown with petty larceny, by saying, ``Now, I sawed you do
it!'' when Everything is capable, with the greatest ease, of being
changed into Anything; and ``Nothing is, but thinking makes it so.''
Now, too, I perceive my first experience of the dreary sensation---%
often to return in after-life---of being unable, next day, to get
back to the dull, settled world; of wanting to live for ever in the
bright atmosphere I have quitted; of doting on the little Fairy,
with the wand like a celestial Barber's Pole, and pining for a Fairy
immortality along with her.  Ah, she comes back, in many shapes, as
my eye wanders down the branches of my Christmas Tree, and goes as
often, and has never yet stayed by me!

Out of this delight springs the toy-theatre,---there it is, with its
familiar proscenium, and ladies in feathers, in the boxes!---and all
its attendant occupation with paste and glue, and gum, and water
colours, in the getting-up of The Miller and his Men, and Elizabeth,
or the Exile of Siberia.  In spite of a few besetting accidents and
failures (particularly an unreasonable disposition in the
respectable Kelmar, and some others, to become faint in the legs,
and double up, at exciting points of the drama), a teeming world of
fancies so suggestive and all-embracing, that, far below it on my
Christmas Tree, I see dark, dirty, real Theatres in the day-time,
adorned with these associations as with the freshest garlands of the
rarest flowers, and charming me yet.

But hark!  The Waits are playing, and they break my childish sleep!
What images do I associate with the Christmas music as I see them
set forth on the Christmas Tree?  Known before all the others,
keeping far apart from all the others, they gather round my little
bed.  An angel, speaking to a group of shepherds in a field; some
travellers, with eyes uplifted, following a star; a baby in a
manger; a child in a spacious temple, talking with grave men; a
solemn figure, with a mild and beautiful face, raising a dead girl
by the hand; again, near a city gate, calling back the son of a
widow, on his bier, to life; a crowd of people looking through the
opened roof of a chamber where he sits, and letting down a sick
person on a bed, with ropes; the same, in a tempest, walking on the
water to a ship; again, on a sea-shore, teaching a great multitude;
again, with a child upon his knee, and other children round; again,
restoring sight to the blind, speech to the dumb, hearing to the
deaf, health to the sick, strength to the lame, knowledge to the
ignorant; again, dying upon a Cross, watched by armed soldiers, a
thick darkness coming on, the earth beginning to shake, and only one
voice heard, ``Forgive them, for they know not what they do.''

Still, on the lower and maturer branches of the Tree, Christmas
associations cluster thick.  School-books shut up; Ovid and Virgil
silenced; the Rule of Three, with its cool impertinent inquiries,
long disposed of; Terence and Plautus acted no more, in an arena of
huddled desks and forms, all chipped, and notched, and inked;
cricket-bats, stumps, and balls, left higher up, with the smell of
trodden grass and the softened noise of shouts in the evening air;
the tree is still fresh, still gay.  If I no more come home at
Christmas-time, there will be boys and girls (thank Heaven!) while
the World lasts; and they do!  Yonder they dance and play upon the
branches of my Tree, God bless them, merrily, and my heart dances
and plays too!

And I do come home at Christmas.  We all do, or we all should.  We
all come home, or ought to come home, for a short holiday---the
longer, the better---from the great boarding-school, where we are for
ever working at our arithmetical slates, to take, and give a rest.
As to going a visiting, where can we not go, if we will; where have
we not been, when we would; starting our fancy from our Christmas
Tree!

Away into the winter prospect.  There are many such upon the tree!
On, by low-lying, misty grounds, through fens and fogs, up long
hills, winding dark as caverns between thick plantations, almost
shutting out the sparkling stars; so, out on broad heights, until we
stop at last, with sudden silence, at an avenue.  The gate-bell has
a deep, half-awful sound in the frosty air; the gate swings open on
its hinges; and, as we drive up to a great house, the glancing
lights grow larger in the windows, and the opposing rows of trees
seem to fall solemnly back on either side, to give us place.  At
intervals, all day, a frightened hare has shot across this whitened
turf; or the distant clatter of a herd of deer trampling the hard
frost, has, for the minute, crushed the silence too.  Their watchful
eyes beneath the fern may be shining now, if we could see them, like
the icy dewdrops on the leaves; but they are still, and all is
still.  And so, the lights growing larger, and the trees falling
back before us, and closing up again behind us, as if to forbid
retreat, we come to the house.

There is probably a smell of roasted chestnuts and other good
comfortable things all the time, for we are telling Winter Stories---%
Ghost Stories, or more shame for us---round the Christmas fire; and
we have never stirred, except to draw a little nearer to it.  But,
no matter for that.  We came to the house, and it is an old house,
full of great chimneys where wood is burnt on ancient dogs upon the
hearth, and grim portraits (some of them with grim legends, too)
lower distrustfully from the oaken panels of the walls.  We are a
middle-aged nobleman, and we make a generous supper with our host
and hostess and their guests---it being Christmas-time, and the old
house full of company---and then we go to bed.  Our room is a very
old room.  It is hung with tapestry.  We don't like the portrait of
a cavalier in green, over the fireplace.  There are great black
beams in the ceiling, and there is a great black bedstead, supported
at the foot by two great black figures, who seem to have come off a
couple of tombs in the old baronial church in the park, for our
particular accommodation.  But, we are not a superstitious nobleman,
and we don't mind.  Well! we dismiss our servant, lock the door, and
sit before the fire in our dressing-gown, musing about a great many
things.  At length we go to bed.  Well! we can't sleep.  We toss and
tumble, and can't sleep.  The embers on the hearth burn fitfully and
make the room look ghostly.  We can't help peeping out over the
counterpane, at the two black figures and the cavalier---that wicked-%
looking cavalier---in green.  In the flickering light they seem to
advance and retire:  which, though we are not by any means a
superstitious nobleman, is not agreeable.  Well! we get nervous---%
more and more nervous.  We say ``This is very foolish, but we can't
stand this; we'll pretend to be ill, and knock up somebody.''  Well!
we are just going to do it, when the locked door opens, and there
comes in a young woman, deadly pale, and with long fair hair, who
glides to the fire, and sits down in the chair we have left there,
wringing her hands.  Then, we notice that her clothes are wet.  Our
tongue cleaves to the roof of our mouth, and we can't speak; but, we
observe her accurately.  Her clothes are wet; her long hair is
dabbled with moist mud; she is dressed in the fashion of two hundred
years ago; and she has at her girdle a bunch of rusty keys.  Well!
there she sits, and we can't even faint, we are in such a state
about it.  Presently she gets up, and tries all the locks in the
room with the rusty keys, which won't fit one of them; then, she
fixes her eyes on the portrait of the cavalier in green, and says,
in a low, terrible voice, ``The stags know it!''  After that, she
wrings her hands again, passes the bedside, and goes out at the
door.  We hurry on our dressing-gown, seize our pistols (we always
travel with pistols), and are following, when we find the door
locked.  We turn the key, look out into the dark gallery; no one
there.  We wander away, and try to find our servant.  Can't be done.
We pace the gallery till daybreak; then return to our deserted room,
fall asleep, and are awakened by our servant (nothing ever haunts
him) and the shining sun.  Well! we make a wretched breakfast, and
all the company say we look queer.  After breakfast, we go over the
house with our host, and then we take him to the portrait of the
cavalier in green, and then it all comes out.  He was false to a
young housekeeper once attached to that family, and famous for her
beauty, who drowned herself in a pond, and whose body was
discovered, after a long time, because the stags refused to drink of
the water.  Since which, it has been whispered that she traverses
the house at midnight (but goes especially to that room where the
cavalier in green was wont to sleep), trying the old locks with the
rusty keys.  Well! we tell our host of what we have seen, and a
shade comes over his features, and he begs it may be hushed up; and
so it is.  But, it's all true; and we said so, before we died (we
are dead now) to many responsible people.

There is no end to the old houses, with resounding galleries, and
dismal state-bedchambers, and haunted wings shut up for many years,
through which we may ramble, with an agreeable creeping up our back,
and encounter any number of ghosts, but (it is worthy of remark
perhaps) reducible to a very few general types and classes; for,
ghosts have little originality, and ``walk'' in a beaten track.  Thus,
it comes to pass, that a certain room in a certain old hall, where a
certain bad lord, baronet, knight, or gentleman, shot himself, has
certain planks in the floor from which the blood \emph{will} \emph{not} be taken
out.  You may scrape and scrape, as the present owner has done, or
plane and plane, as his father did, or scrub and scrub, as his
grandfather did, or burn and burn with strong acids, as his great-%
grandfather did, but, there the blood will still be---no redder and
no paler---no more and no less---always just the same.  Thus, in such
another house there is a haunted door, that never will keep open; or
another door that never will keep shut, or a haunted sound of a
spinning-wheel, or a hammer, or a footstep, or a cry, or a sigh, or
a horse's tramp, or the rattling of a chain.  Or else, there is a
turret-clock, which, at the midnight hour, strikes thirteen when the
head of the family is going to die; or a shadowy, immovable black
carriage which at such a time is always seen by somebody, waiting
near the great gates in the stable-yard.  Or thus, it came to pass
how Lady Mary went to pay a visit at a large wild house in the
Scottish Highlands, and, being fatigued with her long journey,
retired to bed early, and innocently said, next morning, at the
breakfast-table, ``How odd, to have so late a party last night, in
this remote place, and not to tell me of it, before I went to bed!''
Then, every one asked Lady Mary what she meant?  Then, Lady Mary
replied, ``Why, all night long, the carriages were driving round and
round the terrace, underneath my window!''  Then, the owner of the
house turned pale, and so did his Lady, and Charles Macdoodle of
Macdoodle signed to Lady Mary to say no more, and every one was
silent.  After breakfast, Charles Macdoodle told Lady Mary that it
was a tradition in the family that those rumbling carriages on the
terrace betokened death.  And so it proved, for, two months
afterwards, the Lady of the mansion died.  And Lady Mary, who was a
Maid of Honour at Court, often told this story to the old Queen
Charlotte; by this token that the old King always said, ``Eh, eh?
What, what?  Ghosts, ghosts?  No such thing, no such thing!''  And
never left off saying so, until he went to bed.

Or, a friend of somebody's whom most of us know, when he was a young
man at college, had a particular friend, with whom he made the
compact that, if it were possible for the Spirit to return to this
earth after its separation from the body, he of the twain who first
died, should reappear to the other.  In course of time, this compact
was forgotten by our friend; the two young men having progressed in
life, and taken diverging paths that were wide asunder.  But, one
night, many years afterwards, our friend being in the North of
England, and staying for the night in an inn, on the Yorkshire
Moors, happened to look out of bed; and there, in the moonlight,
leaning on a bureau near the window, steadfastly regarding him, saw
his old college friend!  The appearance being solemnly addressed,
replied, in a kind of whisper, but very audibly, ``Do not come near
me.  I am dead.  I am here to redeem my promise.  I come from
another world, but may not disclose its secrets!''  Then, the whole
form becoming paler, melted, as it were, into the moonlight, and
faded away.

Or, there was the daughter of the first occupier of the picturesque
Elizabethan house, so famous in our neighbourhood.  You have heard
about her?  No!  Why, \emph{she} went out one summer evening at twilight,
when she was a beautiful girl, just seventeen years of age, to
gather flowers in the garden; and presently came running, terrified,
into the hall to her father, saying, ``Oh, dear father, I have met
myself!''  He took her in his arms, and told her it was fancy, but
she said, ``Oh no!  I met myself in the broad walk, and I was pale
and gathering withered flowers, and I turned my head, and held them
up!''  And, that night, she died; and a picture of her story was
begun, though never finished, and they say it is somewhere in the
house to this day, with its face to the wall.

Or, the uncle of my brother's wife was riding home on horseback, one
mellow evening at sunset, when, in a green lane close to his own
house, he saw a man standing before him, in the very centre of a
narrow way.  ``Why does that man in the cloak stand there!'' he
thought.  ``Does he want me to ride over him?''  But the figure never
moved.  He felt a strange sensation at seeing it so still, but
slackened his trot and rode forward.  When he was so close to it, as
almost to touch it with his stirrup, his horse shied, and the figure
glided up the bank, in a curious, unearthly manner---backward, and
without seeming to use its feet---and was gone.  The uncle of my
brother's wife, exclaiming, ``Good Heaven!  It's my cousin Harry,
from Bombay!'' put spurs to his horse, which was suddenly in a
profuse sweat, and, wondering at such strange behaviour, dashed
round to the front of his house.  There, he saw the same figure,
just passing in at the long French window of the drawing-room,
opening on the ground.  He threw his bridle to a servant, and
hastened in after it.  His sister was sitting there, alone.  ``Alice,
where's my cousin Harry?''  ``Your cousin Harry, John?''  ``Yes.  From
Bombay.  I met him in the lane just now, and saw him enter here,
this instant.''  Not a creature had been seen by any one; and in that
hour and minute, as it afterwards appeared, this cousin died in
India.

Or, it was a certain sensible old maiden lady, who died at ninety-%
nine, and retained her faculties to the last, who really did see the
Orphan Boy; a story which has often been incorrectly told, but, of
which the real truth is this---because it is, in fact, a story
belonging to our family---and she was a connexion of our family.
When she was about forty years of age, and still an uncommonly fine
woman (her lover died young, which was the reason why she never
married, though she had many offers), she went to stay at a place in
Kent, which her brother, an Indian-Merchant, had newly bought.
There was a story that this place had once been held in trust by the
guardian of a young boy; who was himself the next heir, and who
killed the young boy by harsh and cruel treatment.  She knew nothing
of that.  It has been said that there was a Cage in her bedroom in
which the guardian used to put the boy.  There was no such thing.
There was only a closet.  She went to bed, made no alarm whatever in
the night, and in the morning said composedly to her maid when she
came in, ``Who is the pretty forlorn-looking child who has been
peeping out of that closet all night?''  The maid replied by giving a
loud scream, and instantly decamping.  She was surprised; but she
was a woman of remarkable strength of mind, and she dressed herself
and went downstairs, and closeted herself with her brother.  ``Now,
Walter,'' she said, ``I have been disturbed all night by a pretty,
forlorn-looking boy, who has been constantly peeping out of that
closet in my room, which I can't open.  This is some trick.''  ``I am
afraid not, Charlotte,'' said he, ``for it is the legend of the house.
It is the Orphan Boy.  What did he do?''  ``He opened the door
softly,'' said she, ``and peeped out.  Sometimes, he came a step or
two into the room.  Then, I called to him, to encourage him, and he
shrunk, and shuddered, and crept in again, and shut the door.''  ``The
closet has no communication, Charlotte,'' said her brother, ``with any
other part of the house, and it's nailed up.''  This was undeniably
true, and it took two carpenters a whole forenoon to get it open,
for examination.  Then, she was satisfied that she had seen the
Orphan Boy.  But, the wild and terrible part of the story is, that
he was also seen by three of her brother's sons, in succession, who
all died young.  On the occasion of each child being taken ill, he
came home in a heat, twelve hours before, and said, Oh, Mamma, he
had been playing under a particular oak-tree, in a certain meadow,
with a strange boy---a pretty, forlorn-looking boy, who was very
timid, and made signs!  From fatal experience, the parents came to
know that this was the Orphan Boy, and that the course of that child
whom he chose for his little playmate was surely run.

Legion is the name of the German castles, where we sit up alone to
wait for the Spectre---where we are shown into a room, made
comparatively cheerful for our reception---where we glance round at
the shadows, thrown on the blank walls by the crackling fire---where
we feel very lonely when the village innkeeper and his pretty
daughter have retired, after laying down a fresh store of wood upon
the hearth, and setting forth on the small table such supper-cheer
as a cold roast capon, bread, grapes, and a flask of old Rhine wine---%
where the reverberating doors close on their retreat, one after
another, like so many peals of sullen thunder---and where, about the
small hours of the night, we come into the knowledge of divers
supernatural mysteries.  Legion is the name of the haunted German
students, in whose society we draw yet nearer to the fire, while the
schoolboy in the corner opens his eyes wide and round, and flies off
the footstool he has chosen for his seat, when the door accidentally
blows open.  Vast is the crop of such fruit, shining on our
Christmas Tree; in blossom, almost at the very top; ripening all
down the boughs!

Among the later toys and fancies hanging there---as idle often and
less pure---be the images once associated with the sweet old Waits,
the softened music in the night, ever unalterable!  Encircled by the
social thoughts of Christmas-time, still let the benignant figure of
my childhood stand unchanged!  In every cheerful image and
suggestion that the season brings, may the bright star that rested
above the poor roof, be the star of all the Christian World!  A
moment's pause, O vanishing tree, of which the lower boughs are dark
to me as yet, and let me look once more!  I know there are blank
spaces on thy branches, where eyes that I have loved have shone and
smiled; from which they are departed.  But, far above, I see the
raiser of the dead girl, and the Widow's Son; and God is good!  If
Age be hiding for me in the unseen portion of thy downward growth, O
may I, with a grey head, turn a child's heart to that figure yet,
and a child's trustfulness and confidence!

Now, the tree is decorated with bright merriment, and song, and
dance, and cheerfulness.  And they are welcome.  Innocent and
welcome be they ever held, beneath the branches of the Christmas
Tree, which cast no gloomy shadow!  But, as it sinks into the
ground, I hear a whisper going through the leaves.  ``This, in
commemoration of the law of love and kindness, mercy and compassion.
This, in remembrance of Me!''



\chapter{What Christmas Is As We Grow Older}



Time was, with most of us, when Christmas Day encircling all our
limited world like a magic ring, left nothing out for us to miss or
seek; bound together all our home enjoyments, affections, and hopes;
grouped everything and every one around the Christmas fire; and made
the little picture shining in our bright young eyes, complete.

Time came, perhaps, all so soon, when our thoughts over-leaped that
narrow boundary; when there was some one (very dear, we thought
then, very beautiful, and absolutely perfect) wanting to the fulness
of our happiness; when we were wanting too (or we thought so, which
did just as well) at the Christmas hearth by which that some one
sat; and when we intertwined with every wreath and garland of our
life that some one's name.

That was the time for the bright visionary Christmases which have
long arisen from us to show faintly, after summer rain, in the
palest edges of the rainbow!  That was the time for the beatified
enjoyment of the things that were to be, and never were, and yet the
things that were so real in our resolute hope that it would be hard
to say, now, what realities achieved since, have been stronger!

What!  Did that Christmas never really come when we and the
priceless pearl who was our young choice were received, after the
happiest of totally impossible marriages, by the two united families
previously at daggers---drawn on our account?  When brothers and
sisters-in-law who had always been rather cool to us before our
relationship was effected, perfectly doted on us, and when fathers
and mothers overwhelmed us with unlimited incomes?  Was that
Christmas dinner never really eaten, after which we arose, and
generously and eloquently rendered honour to our late rival, present
in the company, then and there exchanging friendship and
forgiveness, and founding an attachment, not to be surpassed in
Greek or Roman story, which subsisted until death?  Has that same
rival long ceased to care for that same priceless pearl, and married
for money, and become usurious?  Above all, do we really know, now,
that we should probably have been miserable if we had won and worn
the pearl, and that we are better without her?

That Christmas when we had recently achieved so much fame; when we
had been carried in triumph somewhere, for doing something great and
good; when we had won an honoured and ennobled name, and arrived and
were received at home in a shower of tears of joy; is it possible
that \emph{that} Christmas has not come yet?

And is our life here, at the best, so constituted that, pausing as
we advance at such a noticeable mile-stone in the track as this
great birthday, we look back on the things that never were, as
naturally and full as gravely as on the things that have been and
are gone, or have been and still are?  If it be so, and so it seems
to be, must we come to the conclusion that life is little better
than a dream, and little worth the loves and strivings that we crowd
into it?

No!  Far be such miscalled philosophy from us, dear Reader, on
Christmas Day!  Nearer and closer to our hearts be the Christmas
spirit, which is the spirit of active usefulness, perseverance,
cheerful discharge of duty, kindness and forbearance!  It is in the
last virtues especially, that we are, or should be, strengthened by
the unaccomplished visions of our youth; for, who shall say that
they are not our teachers to deal gently even with the impalpable
nothings of the earth!

Therefore, as we grow older, let us be more thankful that the circle
of our Christmas associations and of the lessons that they bring,
expands!  Let us welcome every one of them, and summon them to take
their places by the Christmas hearth.

Welcome, old aspirations, glittering creatures of an ardent fancy,
to your shelter underneath the holly!  We know you, and have not
outlived you yet.  Welcome, old projects and old loves, however
fleeting, to your nooks among the steadier lights that burn around
us.  Welcome, all that was ever real to our hearts; and for the
earnestness that made you real, thanks to Heaven!  Do we build no
Christmas castles in the clouds now?  Let our thoughts, fluttering
like butterflies among these flowers of children, bear witness!
Before this boy, there stretches out a Future, brighter than we ever
looked on in our old romantic time, but bright with honour and with
truth.  Around this little head on which the sunny curls lie heaped,
the graces sport, as prettily, as airily, as when there was no
scythe within the reach of Time to shear away the curls of our
first-love.  Upon another girl's face near it---placider but smiling
bright---a quiet and contented little face, we see Home fairly
written.  Shining from the word, as rays shine from a star, we see
how, when our graves are old, other hopes than ours are young, other
hearts than ours are moved; how other ways are smoothed; how other
happiness blooms, ripens, and decays---no, not decays, for other
homes and other bands of children, not yet in being nor for ages yet
to be, arise, and bloom and ripen to the end of all!

Welcome, everything!  Welcome, alike what has been, and what never
was, and what we hope may be, to your shelter underneath the holly,
to your places round the Christmas fire, where what is sits open-%
hearted!  In yonder shadow, do we see obtruding furtively upon the
blaze, an enemy's face?  By Christmas Day we do forgive him!  If the
injury he has done us may admit of such companionship, let him come
here and take his place.  If otherwise, unhappily, let him go hence,
assured that we will never injure nor accuse him.

On this day we shut out Nothing!

``Pause,'' says a low voice.  ``Nothing?  Think!''

``On Christmas Day, we will shut out from our fireside, Nothing.''

``Not the shadow of a vast City where the withered leaves are lying
deep?'' the voice replies.  ``Not the shadow that darkens the whole
globe?  Not the shadow of the City of the Dead?''

Not even that.  Of all days in the year, we will turn our faces
towards that City upon Christmas Day, and from its silent hosts
bring those we loved, among us.  City of the Dead, in the blessed
name wherein we are gathered together at this time, and in the
Presence that is here among us according to the promise, we will
receive, and not dismiss, thy people who are dear to us!

Yes.  We can look upon these children angels that alight, so
solemnly, so beautifully among the living children by the fire, and
can bear to think how they departed from us.  Entertaining angels
unawares, as the Patriarchs did, the playful children are
unconscious of their guests; but we can see them---can see a radiant
arm around one favourite neck, as if there were a tempting of that
child away.  Among the celestial figures there is one, a poor
misshapen boy on earth, of a glorious beauty now, of whom his dying
mother said it grieved her much to leave him here, alone, for so
many years as it was likely would elapse before he came to her---%
being such a little child.  But he went quickly, and was laid upon
her breast, and in her hand she leads him.

There was a gallant boy, who fell, far away, upon a burning sand
beneath a burning sun, and said, ``Tell them at home, with my last
love, how much I could have wished to kiss them once, but that I
died contented and had done my duty!''  Or there was another, over
whom they read the words, ``Therefore we commit his body to the
deep,'' and so consigned him to the lonely ocean and sailed on.  Or
there was another, who lay down to his rest in the dark shadow of
great forests, and, on earth, awoke no more.  O shall they not, from
sand and sea and forest, be brought home at such a time!

There was a dear girl---almost a woman---never to be one---who made a
mourning Christmas in a house of joy, and went her trackless way to
the silent City.  Do we recollect her, worn out, faintly whispering
what could not be heard, and falling into that last sleep for
weariness?  O look upon her now!  O look upon her beauty, her
serenity, her changeless youth, her happiness!  The daughter of
Jairus was recalled to life, to die; but she, more blest, has heard
the same voice, saying unto her, ``Arise for ever!''

We had a friend who was our friend from early days, with whom we
often pictured the changes that were to come upon our lives, and
merrily imagined how we would speak, and walk, and think, and talk,
when we came to be old.  His destined habitation in the City of the
Dead received him in his prime.  Shall he be shut out from our
Christmas remembrance?  Would his love have so excluded us?  Lost
friend, lost child, lost parent, sister, brother, husband, wife, we
will not so discard you!  You shall hold your cherished places in
our Christmas hearts, and by our Christmas fires; and in the season
of immortal hope, and on the birthday of immortal mercy, we will
shut out Nothing!

The winter sun goes down over town and village; on the sea it makes
a rosy path, as if the Sacred tread were fresh upon the water.  A
few more moments, and it sinks, and night comes on, and lights begin
to sparkle in the prospect.  On the hill-side beyond the
shapelessly-diffused town, and in the quiet keeping of the trees
that gird the village-steeple, remembrances are cut in stone,
planted in common flowers, growing in grass, entwined with lowly
brambles around many a mound of earth.  In town and village, there
are doors and windows closed against the weather, there are flaming
logs heaped high, there are joyful faces, there is healthy music of
voices.  Be all ungentleness and harm excluded from the temples of
the Household Gods, but be those remembrances admitted with tender
encouragement!  They are of the time and all its comforting and
peaceful reassurances; and of the history that re-united even upon
earth the living and the dead; and of the broad beneficence and
goodness that too many men have tried to tear to narrow shreds.



\chapter{The Poor Relation's Story}



He was very reluctant to take precedence of so many respected
members of the family, by beginning the round of stories they were
to relate as they sat in a goodly circle by the Christmas fire; and
he modestly suggested that it would be more correct if ``John our
esteemed host'' (whose health he begged to drink) would have the
kindness to begin.  For as to himself, he said, he was so little
used to lead the way that really--- But as they all cried out here,
that he must begin, and agreed with one voice that he might, could,
would, and should begin, he left off rubbing his hands, and took his
legs out from under his armchair, and did begin.

I have no doubt (said the poor relation) that I shall surprise the
assembled members of our family, and particularly John our esteemed
host to whom we are so much indebted for the great hospitality with
which he has this day entertained us, by the confession I am going
to make.  But, if you do me the honour to be surprised at anything
that falls from a person so unimportant in the family as I am, I can
only say that I shall be scrupulously accurate in all I relate.

I am not what I am supposed to be.  I am quite another thing.
Perhaps before I go further, I had better glance at what I \emph{am}
supposed to be.

It is supposed, unless I mistake---the assembled members of our
family will correct me if I do, which is very likely (here the poor
relation looked mildly about him for contradiction); that I am
nobody's enemy but my own.  That I never met with any particular
success in anything.  That I failed in business because I was
unbusiness-like and credulous---in not being prepared for the
interested designs of my partner.  That I failed in love, because I
was ridiculously trustful---in thinking it impossible that Christiana
could deceive me.  That I failed in my expectations from my uncle
Chill, on account of not being as sharp as he could have wished in
worldly matters.  That, through life, I have been rather put upon
and disappointed in a general way.  That I am at present a bachelor
of between fifty-nine and sixty years of age, living on a limited
income in the form of a quarterly allowance, to which I see that
John our esteemed host wishes me to make no further allusion.

The supposition as to my present pursuits and habits is to the
following effect.

I live in a lodging in the Clapham Road---a very clean back room, in
a very respectable house---where I am expected not to be at home in
the day-time, unless poorly; and which I usually leave in the
morning at nine o'clock, on pretence of going to business.  I take
my breakfast---my roll and butter, and my half-pint of coffee---at the
old-established coffee-shop near Westminster Bridge; and then I go
into the City---I don't know why---and sit in Garraway's Coffee House,
and on 'Change, and walk about, and look into a few offices and
counting-houses where some of my relations or acquaintance are so
good as to tolerate me, and where I stand by the fire if the weather
happens to be cold.  I get through the day in this way until five
o'clock, and then I dine:  at a cost, on the average, of one and
threepence.  Having still a little money to spend on my evening's
entertainment, I look into the old-established coffee-shop as I go
home, and take my cup of tea, and perhaps my bit of toast.  So, as
the large hand of the clock makes its way round to the morning hour
again, I make my way round to the Clapham Road again, and go to bed
when I get to my lodging---fire being expensive, and being objected
to by the family on account of its giving trouble and making a dirt.

Sometimes, one of my relations or acquaintances is so obliging as to
ask me to dinner.  Those are holiday occasions, and then I generally
walk in the Park.  I am a solitary man, and seldom walk with
anybody.  Not that I am avoided because I am shabby; for I am not at
all shabby, having always a very good suit of black on (or rather
Oxford mixture, which has the appearance of black and wears much
better); but I have got into a habit of speaking low, and being
rather silent, and my spirits are not high, and I am sensible that I
am not an attractive companion.

The only exception to this general rule is the child of my first
cousin, Little Frank.  I have a particular affection for that child,
and he takes very kindly to me.  He is a diffident boy by nature;
and in a crowd he is soon run over, as I may say, and forgotten.  He
and I, however, get on exceedingly well.  I have a fancy that the
poor child will in time succeed to my peculiar position in the
family.  We talk but little; still, we understand each other.  We
walk about, hand in hand; and without much speaking he knows what I
mean, and I know what he means.  When he was very little indeed, I
used to take him to the windows of the toy-shops, and show him the
toys inside.  It is surprising how soon he found out that I would
have made him a great many presents if I had been in circumstances
to do it.

Little Frank and I go and look at the outside of the Monument---he is
very fond of the Monument---and at the Bridges, and at all the sights
that are free.  On two of my birthdays, we have dined on e-la-mode
beef, and gone at half-price to the play, and been deeply
interested.  I was once walking with him in Lombard Street, which we
often visit on account of my having mentioned to him that there are
great riches there---he is very fond of Lombard Street---when a
gentleman said to me as he passed by, ``Sir, your little son has
dropped his glove.''  I assure you, if you will excuse my remarking
on so trivial a circumstance, this accidental mention of the child
as mine, quite touched my heart and brought the foolish tears into
my eyes.

When Little Frank is sent to school in the country, I shall be very
much at a loss what to do with myself, but I have the intention of
walking down there once a month and seeing him on a half holiday.  I
am told he will then be at play upon the Heath; and if my visits
should be objected to, as unsettling the child, I can see him from a
distance without his seeing me, and walk back again.  His mother
comes of a highly genteel family, and rather disapproves, I am
aware, of our being too much together.  I know that I am not
calculated to improve his retiring disposition; but I think he would
miss me beyond the feeling of the moment if we were wholly
separated.

When I die in the Clapham Road, I shall not leave much more in this
world than I shall take out of it; but, I happen to have a miniature
of a bright-faced boy, with a curling head, and an open shirt-frill
waving down his bosom (my mother had it taken for me, but I can't
believe that it was ever like), which will be worth nothing to sell,
and which I shall beg may he given to Frank.  I have written my dear
boy a little letter with it, in which I have told him that I felt
very sorry to part from him, though bound to confess that I knew no
reason why I should remain here.  I have given him some short
advice, the best in my power, to take warning of the consequences of
being nobody's enemy but his own; and I have endeavoured to comfort
him for what I fear he will consider a bereavement, by pointing out
to him, that I was only a superfluous something to every one but
him; and that having by some means failed to find a place in this
great assembly, I am better out of it.

Such (said the poor relation, clearing his throat and beginning to
speak a little louder) is the general impression about me.  Now, it
is a remarkable circumstance which forms the aim and purpose of my
story, that this is all wrong.  This is not my life, and these are
not my habits.  I do not even live in the Clapham Road.
Comparatively speaking, I am very seldom there.  I reside, mostly,
in a---I am almost ashamed to say the word, it sounds so full of
pretension---in a Castle.  I do not mean that it is an old baronial
habitation, but still it is a building always known to every one by
the name of a Castle.  In it, I preserve the particulars of my
history; they run thus:

It was when I first took John Spatter (who had been my clerk) into
partnership, and when I was still a young man of not more than five-%
and-twenty, residing in the house of my uncle Chill, from whom I had
considerable expectations, that I ventured to propose to Christiana.
I had loved Christiana a long time.  She was very beautiful, and
very winning in all respects.  I rather mistrusted her widowed
mother, who I feared was of a plotting and mercenary turn of mind;
but, I thought as well of her as I could, for Christiana's sake.  I
never had loved any one but Christiana, and she had been all the
world, and O far more than all the world, to me, from our childhood!

Christiana accepted me with her mother's consent, and I was rendered
very happy indeed.  My life at my uncle Chill's was of a spare dull
kind, and my garret chamber was as dull, and bare, and cold, as an
upper prison room in some stern northern fortress.  But, having
Christiana's love, I wanted nothing upon earth.  I would not have
changed my lot with any human being.

Avarice was, unhappily, my uncle Chill's master-vice.  Though he was
rich, he pinched, and scraped, and clutched, and lived miserably.
As Christiana had no fortune, I was for some time a little fearful
of confessing our engagement to him; but, at length I wrote him a
letter, saying how it all truly was.  I put it into his hand one
night, on going to bed.

As I came down-stairs next morning, shivering in the cold December
air; colder in my uncle's unwarmed house than in the street, where
the winter sun did sometimes shine, and which was at all events
enlivened by cheerful faces and voices passing along; I carried a
heavy heart towards the long, low breakfast-room in which my uncle
sat.  It was a large room with a small fire, and there was a great
bay window in it which the rain had marked in the night as if with
the tears of houseless people.  It stared upon a raw yard, with a
cracked stone pavement, and some rusted iron railings half uprooted,
whence an ugly out-building that had once been a dissecting-room (in
the time of the great surgeon who had mortgaged the house to my
uncle), stared at it.

We rose so early always, that at that time of the year we
breakfasted by candle-light.  When I went into the room, my uncle
was so contracted by the cold, and so huddled together in his chair
behind the one dim candle, that I did not see him until I was close
to the table.

As I held out my hand to him, he caught up his stick (being infirm,
he always walked about the house with a stick), and made a blow at
me, and said, ``You fool!''

``Uncle,'' I returned, ``I didn't expect you to be so angry as this.''
Nor had I expected it, though he was a hard and angry old man.

``You didn't expect!'' said he; ``when did you ever expect?  When did
you ever calculate, or look forward, you contemptible dog?''

``These are hard words, uncle!''

``Hard words?  Feathers, to pelt such an idiot as you with,'' said he.
``Here!  Betsy Snap!  Look at him!''

Betsy Snap was a withered, hard-favoured, yellow old woman---our only
domestic---always employed, at this time of the morning, in rubbing
my uncle's legs.  As my uncle adjured her to look at me, he put his
lean grip on the crown of her head, she kneeling beside him, and
turned her face towards me.  An involuntary thought connecting them
both with the Dissecting Room, as it must often have been in the
surgeon's time, passed across my mind in the midst of my anxiety.

``Look at the snivelling milksop!'' said my uncle.  ``Look at the baby!
This is the gentleman who, people say, is nobody's enemy but his
own.  This is the gentleman who can't say no.  This is the gentleman
who was making such large profits in his business that he must needs
take a partner, t'other day.  This is the gentleman who is going to
marry a wife without a penny, and who falls into the hands of
Jezabels who are speculating on my death!''

I knew, now, how great my uncle's rage was; for nothing short of his
being almost beside himself would have induced him to utter that
concluding word, which he held in such repugnance that it was never
spoken or hinted at before him on any account.

``On my death,'' he repeated, as if he were defying me by defying his
own abhorrence of the word.  ``On my death---death---Death!  But I'll
spoil the speculation.  Eat your last under this roof, you feeble
wretch, and may it choke you!''

You may suppose that I had not much appetite for the breakfast to
which I was bidden in these terms; but, I took my accustomed seat.
I saw that I was repudiated henceforth by my uncle; still I could
bear that very well, possessing Christiana's heart.

He emptied his basin of bread and milk as usual, only that he took
it on his knees with his chair turned away from the table where I
sat.  When he had done, he carefully snuffed out the candle; and the
cold, slate-coloured, miserable day looked in upon us.

``Now, Mr.\ Michael,'' said he, ``before we part, I should like to have
a word with these ladies in your presence.''

``As you will, sir,'' I returned; ``but you deceive yourself, and wrong
us, cruelly, if you suppose that there is any feeling at stake in
this contract but pure, disinterested, faithful love.''

To this, he only replied, ``You lie!'' and not one other word.

We went, through half-thawed snow and half-frozen rain, to the house
where Christiana and her mother lived.  My uncle knew them very
well.  They were sitting at their breakfast, and were surprised to
see us at that hour.

``Your servant, ma'am,'' said my uncle to the mother.  ``You divine the
purpose of my visit, I dare say, ma'am.  I understand there is a
world of pure, disinterested, faithful love cooped up here.  I am
happy to bring it all it wants, to make it complete.  I bring you
your son-in-law, ma'am---and you, your husband, miss.  The gentleman
is a perfect stranger to me, but I wish him joy of his wise
bargain.''

He snarled at me as he went out, and I never saw him again.


It is altogether a mistake (continued the poor relation) to suppose
that my dear Christiana, over-persuaded and influenced by her
mother, married a rich man, the dirt from whose carriage wheels is
often, in these changed times, thrown upon me as she rides by.  No,
no.  She married me.

The way we came to be married rather sooner than we intended, was
this.  I took a frugal lodging and was saving and planning for her
sake, when, one day, she spoke to me with great earnestness, and
said:

``My dear Michael, I have given you my heart.  I have said that I
loved you, and I have pledged myself to be your wife.  I am as much
yours through all changes of good and evil as if we had been married
on the day when such words passed between us.  I know you well, and
know that if we should be separated and our union broken off, your
whole life would be shadowed, and all that might, even now, be
stronger in your character for the conflict with the world would
then be weakened to the shadow of what it is!''

``God help me, Christiana!'' said I.  ``You speak the truth.''

``Michael!'' said she, putting her hand in mine, in all maidenly
devotion, ``let us keep apart no longer.  It is but for me to say
that I can live contented upon such means as you have, and I well
know you are happy.  I say so from my heart.  Strive no more alone;
let us strive together.  My dear Michael, it is not right that I
should keep secret from you what you do not suspect, but what
distresses my whole life.  My mother:  without considering that what
you have lost, you have lost for me, and on the assurance of my
faith:  sets her heart on riches, and urges another suit upon me, to
my misery.  I cannot bear this, for to bear it is to be untrue to
you.  I would rather share your struggles than look on.  I want no
better home than you can give me.  I know that you will aspire and
labour with a higher courage if I am wholly yours, and let it be so
when you will!''

I was blest indeed, that day, and a new world opened to me.  We were
married in a very little while, and I took my wife to our happy
home.  That was the beginning of the residence I have spoken of; the
Castle we have ever since inhabited together, dates from that time.
All our children have been born in it.  Our first child---now
married---was a little girl, whom we called Christiana.  Her son is
so like Little Frank, that I hardly know which is which.


The current impression as to my partner's dealings with me is also
quite erroneous.  He did not begin to treat me coldly, as a poor
simpleton, when my uncle and I so fatally quarrelled; nor did he
afterwards gradually possess himself of our business and edge me
out.  On the contrary, he behaved to me with the utmost good faith
and honour.

Matters between us took this turn:- On the day of my separation from
my uncle, and even before the arrival at our counting-house of my
trunks (which he sent after me, \emph{not} carriage paid), I went down to
our room of business, on our little wharf, overlooking the river;
and there I told John Spatter what had happened.  John did not say,
in reply, that rich old relatives were palpable facts, and that love
and sentiment were moonshine and fiction.  He addressed me thus:

``Michael,'' said John, ``we were at school together, and I generally
had the knack of getting on better than you, and making a higher
reputation.''

``You had, John,'' I returned.

``Although'' said John, ``I borrowed your books and lost them; borrowed
your pocket-money, and never repaid it; got you to buy my damaged
knives at a higher price than I had given for them new; and to own
to the windows that I had broken.''

``All not worth mentioning, John Spatter,'' said I, ``but certainly
true.''

``When you were first established in this infant business, which
promises to thrive so well,'' pursued John, ``I came to you, in my
search for almost any employment, and you made me your clerk.''

``Still not worth mentioning, my dear John Spatter,'' said I; ``still,
equally true.''

``And finding that I had a good head for business, and that I was
really useful \emph{to} the business, you did not like to retain me in that
capacity, and thought it an act of justice soon to make me your
partner.''

``Still less worth mentioning than any of those other little
circumstances you have recalled, John Spatter,'' said I; ``for I was,
and am, sensible of your merits and my deficiencies.''

``Now, my good friend,'' said John, drawing my arm through his, as he
had had a habit of doing at school; while two vessels outside the
windows of our counting-house---which were shaped like the stern
windows of a ship---went lightly down the river with the tide, as
John and I might then be sailing away in company, and in trust and
confidence, on our voyage of life; ``let there, under these friendly
circumstances, be a right understanding between us.  You are too
easy, Michael.  You are nobody's enemy but your own.  If I were to
give you that damaging character among our connexion, with a shrug,
and a shake of the head, and a sigh; and if I were further to abuse
the trust you place in me---''

``But you never will abuse it at all, John,'' I observed.

``Never!'' said he; ``but I am putting a case---I say, and if I were
further to abuse that trust by keeping this piece of our common
affairs in the dark, and this other piece in the light, and again
this other piece in the twilight, and so on, I should strengthen my
strength, and weaken your weakness, day by day, until at last I
found myself on the high road to fortune, and you left behind on
some bare common, a hopeless number of miles out of the way.''

``Exactly so,'' said I.

``To prevent this, Michael,'' said John Spatter, ``or the remotest
chance of this, there must be perfect openness between us.  Nothing
must be concealed, and we must have but one interest.''

``My dear John Spatter,'' I assured him, ``that is precisely what I
mean.''

``And when you are too easy,'' pursued John, his face glowing with
friendship, ``you must allow me to prevent that imperfection in your
nature from being taken advantage of, by any one; you must not
expect me to humour it---''

``My dear John Spatter,'' I interrupted, ``I \emph{don't} expect you to humour
it.  I want to correct it.''

``And I, too,'' said John.

``Exactly so!'' cried I.  ``We both have the same end in view; and,
honourably seeking it, and fully trusting one another, and having
but one interest, ours will be a prosperous and happy partnership.''

``I am sure of it!'' returned John Spatter.  And we shook hands most
affectionately.

I took John home to my Castle, and we had a very happy day.  Our
partnership throve well.  My friend and partner supplied what I
wanted, as I had foreseen that he would, and by improving both the
business and myself, amply acknowledged any little rise in life to
which I had helped him.


I am not (said the poor relation, looking at the fire as he slowly
rubbed his hands) very rich, for I never cared to be that; but I
have enough, and am above all moderate wants and anxieties.  My
Castle is not a splendid place, but it is very comfortable, and it
has a warm and cheerful air, and is quite a picture of Home.

Our eldest girl, who is very like her mother, married John Spatter's
eldest son.  Our two families are closely united in other ties of
attachment.  It is very pleasant of an evening, when we are all
assembled together---which frequently happens---and when John and I
talk over old times, and the one interest there has always been
between us.

I really do not know, in my Castle, what loneliness is.  Some of our
children or grandchildren are always about it, and the young voices
of my descendants are delightful---O, how delightful!---to me to hear.
My dearest and most devoted wife, ever faithful, ever loving, ever
helpful and sustaining and consoling, is the priceless blessing of
my house; from whom all its other blessings spring.  We are rather a
musical family, and when Christiana sees me, at any time, a little
weary or depressed, she steals to the piano and sings a gentle air
she used to sing when we were first betrothed.  So weak a man am I,
that I cannot bear to hear it from any other source.  They played it
once, at the Theatre, when I was there with Little Frank; and the
child said wondering, ``Cousin Michael, whose hot tears are these
that have fallen on my hand!''

Such is my Castle, and such are the real particulars of my life
therein preserved.  I often take Little Frank home there.  He is
very welcome to my grandchildren, and they play together.  At this
time of the year---the Christmas and New Year time---I am seldom out
of my Castle.  For, the associations of the season seem to hold me
there, and the precepts of the season seem to teach me that it is
well to be there.


``And the Castle is---'' observed a grave, kind voice among the
company.

``Yes.  My Castle,'' said the poor relation, shaking his head as he
still looked at the fire, ``is in the Air.  John our esteemed host
suggests its situation accurately.  My Castle is in the Air!  I have
done.  Will you be so good as to pass the story?''



\chapter{The Child's Story}



Once upon a time, a good many years ago, there was a traveller, and
he set out upon a journey.  It was a magic journey, and was to seem
very long when he began it, and very short when he got half way
through.

He travelled along a rather dark path for some little time, without
meeting anything, until at last he came to a beautiful child.  So he
said to the child, ``What do you do here?''  And the child said, ``I am
always at play.  Come and play with me!''

So, he played with that child, the whole day long, and they were
very merry.  The sky was so blue, the sun was so bright, the water
was so sparkling, the leaves were so green, the flowers were so
lovely, and they heard such singing-birds and saw so many butteries,
that everything was beautiful.  This was in fine weather.  When it
rained, they loved to watch the falling drops, and to smell the
fresh scents.  When it blew, it was delightful to listen to the
wind, and fancy what it said, as it came rushing from its home---%
where was that, they wondered!---whistling and howling, driving the
clouds before it, bending the trees, rumbling in the chimneys,
shaking the house, and making the sea roar in fury.  But, when it
snowed, that was best of all; for, they liked nothing so well as to
look up at the white flakes falling fast and thick, like down from
the breasts of millions of white birds; and to see how smooth and
deep the drift was; and to listen to the hush upon the paths and
roads.

They had plenty of the finest toys in the world, and the most
astonishing picture-books:  all about scimitars and slippers and
turbans, and dwarfs and giants and genii and fairies, and blue-%
beards and bean-stalks and riches and caverns and forests and
Valentines and Orsons:  and all new and all true.

But, one day, of a sudden, the traveller lost the child.  He called
to him over and over again, but got no answer.  So, he went upon his
road, and went on for a little while without meeting anything, until
at last he came to a handsome boy.  So, he said to the boy, ``What do
you do here?''  And the boy said, ``I am always learning.  Come and
learn with me.''

So he learned with that boy about Jupiter and Juno, and the Greeks
and the Romans, and I don't know what, and learned more than I could
tell---or he either, for he soon forgot a great deal of it.  But,
they were not always learning; they had the merriest games that ever
were played.  They rowed upon the river in summer, and skated on the
ice in winter; they were active afoot, and active on horseback; at
cricket, and all games at ball; at prisoner's base, hare and hounds,
follow my leader, and more sports than I can think of; nobody could
beat them.  They had holidays too, and Twelfth cakes, and parties
where they danced till midnight, and real Theatres where they saw
palaces of real gold and silver rise out of the real earth, and saw
all the wonders of the world at once.  As to friends, they had such
dear friends and so many of them, that I want the time to reckon
them up.  They were all young, like the handsome boy, and were never
to be strange to one another all their lives through.

Still, one day, in the midst of all these pleasures, the traveller
lost the boy as he had lost the child, and, after calling to him in
vain, went on upon his journey.  So he went on for a little while
without seeing anything, until at last he came to a young man.  So,
he said to the young man, ``What do you do here?''  And the young man
said, ``I am always in love.  Come and love with me.''

So, he went away with that young man, and presently they came to one
of the prettiest girls that ever was seen---just like Fanny in the
corner there---and she had eyes like Fanny, and hair like Fanny, and
dimples like Fanny's, and she laughed and coloured just as Fanny
does while I am talking about her.  So, the young man fell in love
directly---just as Somebody I won't mention, the first time he came
here, did with Fanny.  Well! he was teased sometimes---just as
Somebody used to be by Fanny; and they quarrelled sometimes---just as
Somebody and Fanny used to quarrel; and they made it up, and sat in
the dark, and wrote letters every day, and never were happy asunder,
and were always looking out for one another and pretending not to,
and were engaged at Christmas-time, and sat close to one another by
the fire, and were going to be married very soon---all exactly like
Somebody I won't mention, and Fanny!

But, the traveller lost them one day, as he had lost the rest of his
friends, and, after calling to them to come back, which they never
did, went on upon his journey.  So, he went on for a little while
without seeing anything, until at last he came to a middle-aged
gentleman.  So, he said to the gentleman, ``What are you doing here?''
And his answer was, ``I am always busy.  Come and be busy with me!''

So, he began to be very busy with that gentleman, and they went on
through the wood together.  The whole journey was through a wood,
only it had been open and green at first, like a wood in spring; and
now began to be thick and dark, like a wood in summer; some of the
little trees that had come out earliest, were even turning brown.
The gentleman was not alone, but had a lady of about the same age
with him, who was his Wife; and they had children, who were with
them too.  So, they all went on together through the wood, cutting
down the trees, and making a path through the branches and the
fallen leaves, and carrying burdens, and working hard.

Sometimes, they came to a long green avenue that opened into deeper
woods.  Then they would hear a very little, distant voice crying,
``Father, father, I am another child!  Stop for me!''  And presently
they would see a very little figure, growing larger as it came
along, running to join them.  When it came up, they all crowded
round it, and kissed and welcomed it; and then they all went on
together.

Sometimes, they came to several avenues at once, and then they all
stood still, and one of the children said, ``Father, I am going to
sea,'' and another said, ``Father, I am going to India,'' and another,
``Father, I am going to seek my fortune where I can,'' and another,
``Father, I am going to Heaven!''  So, with many tears at parting,
they went, solitary, down those avenues, each child upon its way;
and the child who went to Heaven, rose into the golden air and
vanished.

Whenever these partings happened, the traveller looked at the
gentleman, and saw him glance up at the sky above the trees, where
the day was beginning to decline, and the sunset to come on.  He
saw, too, that his hair was turning grey.  But, they never could
rest long, for they had their journey to perform, and it was
necessary for them to be always busy.

At last, there had been so many partings that there were no children
left, and only the traveller, the gentleman, and the lady, went upon
their way in company.  And now the wood was yellow; and now brown;
and the leaves, even of the forest trees, began to fall.

So, they came to an avenue that was darker than the rest, and were
pressing forward on their journey without looking down it when the
lady stopped.

``My husband,'' said the lady.  ``I am called.''

They listened, and they heard a voice a long way down the avenue,
say, ``Mother, mother!''

It was the voice of the first child who had said, ``I am going to
Heaven!'' and the father said, ``I pray not yet.  The sunset is very
near.  I pray not yet!''

But, the voice cried, ``Mother, mother!'' without minding him, though
his hair was now quite white, and tears were on his face.

Then, the mother, who was already drawn into the shade of the dark
avenue and moving away with her arms still round his neck, kissed
him, and said, ``My dearest, I am summoned, and I go!''  And she was
gone.  And the traveller and he were left alone together.

And they went on and on together, until they came to very near the
end of the wood:  so near, that they could see the sunset shining
red before them through the trees.

Yet, once more, while he broke his way among the branches, the
traveller lost his friend.  He called and called, but there was no
reply, and when he passed out of the wood, and saw the peaceful sun
going down upon a wide purple prospect, he came to an old man
sitting on a fallen tree.  So, he said to the old man, ``What do you
do here?''  And the old man said with a calm smile, ``I am always
remembering.  Come and remember with me!''

So the traveller sat down by the side of that old man, face to face
with the serene sunset; and all his friends came softly back and
stood around him.  The beautiful child, the handsome boy, the young
man in love, the father, mother, and children:  every one of them
was there, and he had lost nothing.  So, he loved them all, and was
kind and forbearing with them all, and was always pleased to watch
them all, and they all honoured and loved him.  And I think the
traveller must be yourself, dear Grandfather, because this what you
do to us, and what we do to you.



\chapter{The Schoolboy's Story}



Being rather young at present---I am getting on in years, but still I
am rather young---I have no particular adventures of my own to fall
back upon.  It wouldn't much interest anybody here, I suppose, to
know what a screw the Reverend is, or what a griffin \emph{she} is, or how
they do stick it into parents---particularly hair-cutting, and
medical attendance.  One of our fellows was charged in his half's
account twelve and sixpence for two pills---tolerably profitable at
six and threepence a-piece, I should think---and he never took them
either, but put them up the sleeve of his jacket.

As to the beef, it's shameful.  It's \emph{not} beef.  Regular beef isn't
veins.  You can chew regular beef.  Besides which, there's gravy to
regular beef, and you never see a drop to ours.  Another of our
fellows went home ill, and heard the family doctor tell his father
that he couldn't account for his complaint unless it was the beer.
Of course it was the beer, and well it might be!

However, beef and Old Cheeseman are two different things.  So is
beer.  It was Old Cheeseman I meant to tell about; not the manner in
which our fellows get their constitutions destroyed for the sake of
profit.

Why, look at the pie-crust alone.  There's no flakiness in it.  It's
solid---like damp lead.  Then our fellows get nightmares, and are
bolstered for calling out and waking other fellows.  Who can wonder!

Old Cheeseman one night walked in his sleep, put his hat on over his
night-cap, got hold of a fishing-rod and a cricket-bat, and went
down into the parlour, where they naturally thought from his
appearance he was a Ghost.  Why, he never would have done that if
his meals had been wholesome.  When we all begin to walk in our
sleeps, I suppose they'll be sorry for it.

Old Cheeseman wasn't second Latin Master then; he was a fellow
himself.  He was first brought there, very small, in a post-chaise,
by a woman who was always taking snuff and shaking him---and that was
the most he remembered about it.  He never went home for the
holidays.  His accounts (he never learnt any extras) were sent to a
Bank, and the Bank paid them; and he had a brown suit twice a-year,
and went into boots at twelve.  They were always too big for him,
too.

In the Midsummer holidays, some of our fellows who lived within
walking distance, used to come back and climb the trees outside the
playground wall, on purpose to look at Old Cheeseman reading there
by himself.  He was always as mild as the tea---and \emph{that}'S pretty
mild, I should hope!---so when they whistled to him, he looked up and
nodded; and when they said, ``Halloa, Old Cheeseman, what have you
had for dinner?'' he said, ``Boiled mutton;'' and when they said, ``An't
it solitary, Old Cheeseman?'' he said, ``It is a little dull
sometimes:'' and then they said, ``Well good-bye, Old Cheeseman!'' and
climbed down again.  Of course it was imposing on Old Cheeseman to
give him nothing but boiled mutton through a whole Vacation, but
that was just like the system.  When they didn't give him boiled
mutton, they gave him rice pudding, pretending it was a treat.  And
saved the butcher.

So Old Cheeseman went on.  The holidays brought him into other
trouble besides the loneliness; because when the fellows began to
come back, not wanting to, he was always glad to see them; which was
aggravating when they were not at all glad to see him, and so he got
his head knocked against walls, and that was the way his nose bled.
But he was a favourite in general.  Once a subscription was raised
for him; and, to keep up his spirits, he was presented before the
holidays with two white mice, a rabbit, a pigeon, and a beautiful
puppy.  Old Cheeseman cried about it---especially soon afterwards,
when they all ate one another.

Of course Old Cheeseman used to be called by the names of all sorts
of cheeses---Double Glo'sterman, Family Cheshireman, Dutchman, North
Wiltshireman, and all that.  But he never minded it.  And I don't
mean to say he was old in point of years---because he wasn't---only he
was called from the first, Old Cheeseman.

At last, Old Cheeseman was made second Latin Master.  He was brought
in one morning at the beginning of a new half, and presented to the
school in that capacity as ``Mr.\ Cheeseman.''  Then our fellows all
agreed that Old Cheeseman was a spy, and a deserter, who had gone
over to the enemy's camp, and sold himself for gold.  It was no
excuse for him that he had sold himself for very little gold---two
pound ten a quarter and his washing, as was reported.  It was
decided by a Parliament which sat about it, that Old Cheeseman's
mercenary motives could alone be taken into account, and that he had
``coined our blood for drachmas.''  The Parliament took the expression
out of the quarrel scene between Brutus and Cassius.

When it was settled in this strong way that Old Cheeseman was a
tremendous traitor, who had wormed himself into our fellows' secrets
on purpose to get himself into favour by giving up everything he
knew, all courageous fellows were invited to come forward and enrol
themselves in a Society for making a set against him.  The President
of the Society was First boy, named Bob Tarter.  His father was in
the West Indies, and he owned, himself, that his father was worth
Millions.  He had great power among our fellows, and he wrote a
parody, beginning -%

\begin{verse}
  ``Who made believe to be so meek\\
  That we could hardly hear him speak,\\
  Yet turned out an Informing Sneak?\\
  Old Cheeseman.''
\end{verse}

- and on in that way through more than a dozen verses, which he used
to go and sing, every morning, close by the new master's desk.  He
trained one of the low boys, too, a rosy-cheeked little Brass who
didn't care what he did, to go up to him with his Latin Grammar one
morning, and say it so:  \emph{nominativus} \emph{pronominum}---Old Cheeseman, \emph{raro}
\emph{exprimitur}---was never suspected, \emph{nisi} \emph{distinctionis}---of being an
informer, \emph{aut} \emph{emphasis} \emph{gratia}---until he proved one.  \emph{ut}---for
instance, \emph{vos} \emph{damnastis}---when he sold the boys.  \emph{quasi}---as though,
\emph{dicat}---he should say, \emph{pretaerea} \emph{nemo}---I'm a Judas!  All this
produced a great effect on Old Cheeseman.  He had never had much
hair; but what he had, began to get thinner and thinner every day.
He grew paler and more worn; and sometimes of an evening he was seen
sitting at his desk with a precious long snuff to his candle, and
his hands before his face, crying.  But no member of the Society
could pity him, even if he felt inclined, because the President said
it was Old Cheeseman's conscience.

So Old Cheeseman went on, and didn't he lead a miserable life!  Of
course the Reverend turned up his nose at him, and of course \emph{she}
did---because both of them always do that at all the masters---but he
suffered from the fellows most, and he suffered from them
constantly.  He never told about it, that the Society could find
out; but he got no credit for that, because the President said it
was Old Cheeseman's cowardice.

He had only one friend in the world, and that one was almost as
powerless as he was, for it was only Jane.  Jane was a sort of
wardrobe woman to our fellows, and took care of the boxes.  She had
come at first, I believe, as a kind of apprentice---some of our
fellows say from a Charity, but I don't know---and after her time was
out, had stopped at so much a year.  So little a year, perhaps I
ought to say, for it is far more likely.  However, she had put some
pounds in the Savings' Bank, and she was a very nice young woman.
She was not quite pretty; but she had a very frank, honest, bright
face, and all our fellows were fond of her.  She was uncommonly neat
and cheerful, and uncommonly comfortable and kind.  And if anything
was the matter with a fellow's mother, he always went and showed the
letter to Jane.

Jane was Old Cheeseman's friend.  The more the Society went against
him, the more Jane stood by him.  She used to give him a good-%
humoured look out of her still-room window, sometimes, that seemed
to set him up for the day.  She used to pass out of the orchard and
the kitchen garden (always kept locked, I believe you!) through the
playground, when she might have gone the other way, only to give a
turn of her head, as much as to say ``Keep up your spirits!'' to Old
Cheeseman.  His slip of a room was so fresh and orderly that it was
well known who looked after it while he was at his desk; and when
our fellows saw a smoking hot dumpling on his plate at dinner, they
knew with indignation who had sent it up.

Under these circumstances, the Society resolved, after a quantity of
meeting and debating, that Jane should be requested to cut Old
Cheeseman dead; and that if she refused, she must be sent to
Coventry herself.  So a deputation, headed by the President, was
appointed to wait on Jane, and inform her of the vote the Society
had been under the painful necessity of passing.  She was very much
respected for all her good qualities, and there was a story about
her having once waylaid the Reverend in his own study, and got a
fellow off from severe punishment, of her own kind comfortable
heart.  So the deputation didn't much like the job.  However, they
went up, and the President told Jane all about it.  Upon which Jane
turned very red, burst into tears, informed the President and the
deputation, in a way not at all like her usual way, that they were a
parcel of malicious young savages, and turned the whole respected
body out of the room.  Consequently it was entered in the Society's
book (kept in astronomical cypher for fear of detection), that all
communication with Jane was interdicted:  and the President
addressed the members on this convincing instance of Old Cheeseman's
undermining.

But Jane was as true to Old Cheeseman as Old Cheeseman was false to
our fellows---in their opinion, at all events---and steadily continued
to be his only friend.  It was a great exasperation to the Society,
because Jane was as much a loss to them as she was a gain to him;
and being more inveterate against him than ever, they treated him
worse than ever.  At last, one morning, his desk stood empty, his
room was peeped into, and found to be vacant, and a whisper went
about among the pale faces of our fellows that Old Cheeseman, unable
to bear it any longer, had got up early and drowned himself.

The mysterious looks of the other masters after breakfast, and the
evident fact that old Cheeseman was not expected, confirmed the
Society in this opinion.  Some began to discuss whether the
President was liable to hanging or only transportation for life, and
the President's face showed a great anxiety to know which.  However,
he said that a jury of his country should find him game; and that in
his address he should put it to them to lay their hands upon their
hearts and say whether they as Britons approved of informers, and
how they thought they would like it themselves.  Some of the Society
considered that he had better run away until he found a forest where
he might change clothes with a wood-cutter, and stain his face with
blackberries; but the majority believed that if he stood his ground,
his father---belonging as he did to the West Indies, and being worth
millions---could buy him off.

All our fellows' hearts beat fast when the Reverend came in, and
made a sort of a Roman, or a Field Marshal, of himself with the
ruler; as he always did before delivering an address.  But their
fears were nothing to their astonishment when he came out with the
story that Old Cheeseman, ``so long our respected friend and fellow-%
pilgrim in the pleasant plains of knowledge,'' he called him---O yes!
I dare say!  Much of that!---was the orphan child of a disinherited
young lady who had married against her father's wish, and whose
young husband had died, and who had died of sorrow herself, and
whose unfortunate baby (Old Cheeseman) had been brought up at the
cost of a grandfather who would never consent to see it, baby, boy,
or man:  which grandfather was now dead, and serve him right---that's
my putting in---and which grandfather's large property, there being
no will, was now, and all of a sudden and for ever, Old Cheeseman's!
Our so long respected friend and fellow-pilgrim in the pleasant
plains of knowledge, the Reverend wound up a lot of bothering
quotations by saying, would ``come among us once more'' that day
fortnight, when he desired to take leave of us himself, in a more
particular manner.  With these words, he stared severely round at
our fellows, and went solemnly out.

There was precious consternation among the members of the Society,
now.  Lots of them wanted to resign, and lots more began to try to
make out that they had never belonged to it.  However, the President
stuck up, and said that they must stand or fall together, and that
if a breach was made it should be over his body---which was meant to
encourage the Society:  but it didn't.  The President further said,
he would consider the position in which they stood, and would give
them his best opinion and advice in a few days.  This was eagerly
looked for, as he knew a good deal of the world on account of his
father's being in the West Indies.

After days and days of hard thinking, and drawing armies all over
his slate, the President called our fellows together, and made the
matter clear.  He said it was plain that when Old Cheeseman came on
the appointed day, his first revenge would be to impeach the
Society, and have it flogged all round.  After witnessing with joy
the torture of his enemies, and gloating over the cries which agony
would extort from them, the probability was that he would invite the
Reverend, on pretence of conversation, into a private room---say the
parlour into which Parents were shown, where the two great globes
were which were never used---and would there reproach him with the
various frauds and oppressions he had endured at his hands.  At the
close of his observations he would make a signal to a Prizefighter
concealed in the passage, who would then appear and pitch into the
Reverend, till he was left insensible.  Old Cheeseman would then
make Jane a present of from five to ten pounds, and would leave the
establishment in fiendish triumph.

The President explained that against the parlour part, or the Jane
part, of these arrangements he had nothing to say; but, on the part
of the Society, he counselled deadly resistance.  With this view he
recommended that all available desks should be filled with stones,
and that the first word of the complaint should be the signal to
every fellow to let fly at Old Cheeseman.  The bold advice put the
Society in better spirits, and was unanimously taken.  A post about
Old Cheeseman's size was put up in the playground, and all our
fellows practised at it till it was dinted all over.

When the day came, and Places were called, every fellow sat down in
a tremble.  There had been much discussing and disputing as to how
Old Cheeseman would come; but it was the general opinion that he
would appear in a sort of triumphal car drawn by four horses, with
two livery servants in front, and the Prizefighter in disguise up
behind.  So, all our fellows sat listening for the sound of wheels.
But no wheels were heard, for Old Cheeseman walked after all, and
came into the school without any preparation.  Pretty much as he
used to be, only dressed in black.

``Gentlemen,'' said the Reverend, presenting him, ``our so long
respected friend and fellow-pilgrim in the pleasant plains of
knowledge, is desirous to offer a word or two.  Attention,
gentlemen, one and all!''

Every fellow stole his hand into his desk and looked at the
President.  The President was all ready, and taking aim at old
Cheeseman with his eyes.

What did Old Cheeseman then, but walk up to his old desk, look round
him with a queer smile as if there was a tear in his eye, and begin
in a quavering, mild voice, ``My dear companions and old friends!''

Every fellow's hand came out of his desk, and the President suddenly
began to cry.

``My dear companions and old friends,'' said Old Cheeseman, ``you have
heard of my good fortune.  I have passed so many years under this
roof---my entire life so far, I may say---that I hope you have been
glad to hear of it for my sake.  I could never enjoy it without
exchanging congratulations with you.  If we have ever misunderstood
one another at all, pray, my dear boys, let us forgive and forget.
I have a great tenderness for you, and I am sure you return it.  I
want in the fulness of a grateful heart to shake hands with you
every one.  I have come back to do it, if you please, my dear boys.''

Since the President had begun to cry, several other fellows had
broken out here and there:  but now, when Old Cheeseman began with
him as first boy, laid his left hand affectionately on his shoulder
and gave him his right; and when the President said ``Indeed, I don't
deserve it, sir; upon my honour I don't;'' there was sobbing and
crying all over the school.  Every other fellow said he didn't
deserve it, much in the same way; but Old Cheeseman, not minding
that a bit, went cheerfully round to every boy, and wound up with
every master---finishing off the Reverend last.

Then a snivelling little chap in a corner, who was always under some
punishment or other, set up a shrill cry of ``Success to Old
Cheeseman!  Hooray!''  The Reverend glared upon him, and said, ``\emph{Mr}.
Cheeseman, sir.''  But, Old Cheeseman protesting that he liked his
old name a great deal better than his new one, all our fellows took
up the cry; and, for I don't know how many minutes, there was such a
thundering of feet and hands, and such a roaring of Old Cheeseman,
as never was heard.

After that, there was a spread in the dining-room of the most
magnificent kind.  Fowls, tongues, preserves, fruits,
confectionaries, jellies, neguses, barley-sugar temples, trifles,
crackers---eat all you can and pocket what you like---all at Old
Cheeseman's expense.  After that, speeches, whole holiday, double
and treble sets of all manners of things for all manners of games,
donkeys, pony-chaises and drive yourself, dinner for all the masters
at the Seven Bells (twenty pounds a-head our fellows estimated it
at), an annual holiday and feast fixed for that day every year, and
another on Old Cheeseman's birthday---Reverend bound down before the
fellows to allow it, so that he could never back out---all at Old
Cheeseman's expense.

And didn't our fellows go down in a body and cheer outside the Seven
Bells?  O no!

But there's something else besides.  Don't look at the next story-%
teller, for there's more yet.  Next day, it was resolved that the
Society should make it up with Jane, and then be dissolved.  What do
you think of Jane being gone, though!  ``What?  Gone for ever?'' said
our fellows, with long faces.  ``Yes, to be sure,'' was all the answer
they could get.  None of the people about the house would say
anything more.  At length, the first boy took upon himself to ask
the Reverend whether our old friend Jane was really gone?  The
Reverend (he has got a daughter at home---turn-up nose, and red)
replied severely, ``Yes, sir, Miss Pitt is gone.''  The idea of
calling Jane, Miss Pitt!  Some said she had been sent away in
disgrace for taking money from Old Cheeseman; others said she had
gone into Old Cheeseman's service at a rise of ten pounds a year.
All that our fellows knew, was, she was gone.

It was two or three months afterwards, when, one afternoon, an open
carriage stopped at the cricket field, just outside bounds, with a
lady and gentleman in it, who looked at the game a long time and
stood up to see it played.  Nobody thought much about them, until
the same little snivelling chap came in, against all rules, from the
post where he was Scout, and said, ``It's Jane!''  Both Elevens forgot
the game directly, and ran crowding round the carriage.  It \emph{was}
Jane!  In such a bonnet!  And if you'll believe me, Jane was married
to Old Cheeseman.

It soon became quite a regular thing when our fellows were hard at
it in the playground, to see a carriage at the low part of the wall
where it joins the high part, and a lady and gentleman standing up
in it, looking over.  The gentleman was always Old Cheeseman, and
the lady was always Jane.

The first time I ever saw them, I saw them in that way.  There had
been a good many changes among our fellows then, and it had turned
out that Bob Tarter's father wasn't worth Millions!  He wasn't worth
anything.  Bob had gone for a soldier, and Old Cheeseman had
purchased his discharge.  But that's not the carriage.  The carriage
stopped, and all our fellows stopped as soon as it was seen.

``So you have never sent me to Coventry after all!'' said the lady,
laughing, as our fellows swarmed up the wall to shake hands with
her.  ``Are you never going to do it?''

``Never! never! never!'' on all sides.

I didn't understand what she meant then, but of course I do now.  I
was very much pleased with her face though, and with her good way,
and I couldn't help looking at her---and at him too---with all our
fellows clustering so joyfully about them.

They soon took notice of me as a new boy, so I thought I might as
well swarm up the wall myself, and shake hands with them as the rest
did.  I was quite as glad to see them as the rest were, and was
quite as familiar with them in a moment.

``Only a fortnight now,'' said Old Cheeseman, ``to the holidays.  Who
stops?  Anybody?''

A good many fingers pointed at me, and a good many voices cried ``He
does!''  For it was the year when you were all away; and rather low I
was about it, I can tell you.

``Oh!'' said Old Cheeseman.  ``But it's solitary here in the holiday
time.  He had better come to us.''

So I went to their delightful house, and was as happy as I could
possibly be.  They understand how to conduct themselves towards
boys, \emph{they} do.  When they take a boy to the play, for instance, they
\emph{do} take him.  They don't go in after it's begun, or come out before
it's over.  They know how to bring a boy up, too.  Look at their
own!  Though he is very little as yet, what a capital boy he is!
Why, my next favourite to Mrs.\ Cheeseman and Old Cheeseman, is young
Cheeseman.

So, now I have told you all I know about Old Cheeseman.  And it's
not much after all, I am afraid.  Is it?



\chapter{Nobody's Story}



He lived on the bank of a mighty river, broad and deep, which was
always silently rolling on to a vast undiscovered ocean.  It had
rolled on, ever since the world began.  It had changed its course
sometimes, and turned into new channels, leaving its old ways dry
and barren; but it had ever been upon the flow, and ever was to flow
until Time should be no more.  Against its strong, unfathomable
stream, nothing made head.  No living creature, no flower, no leaf,
no particle of animate or inanimate existence, ever strayed back
from the undiscovered ocean.  The tide of the river set resistlessly
towards it; and the tide never stopped, any more than the earth
stops in its circling round the sun.

He lived in a busy place, and he worked very hard to live.  He had
no hope of ever being rich enough to live a month without hard work,
but he was quite content, \emph{God} knows, to labour with a cheerful will.
He was one of an immense family, all of whose sons and daughters
gained their daily bread by daily work, prolonged from their rising
up betimes until their lying down at night.  Beyond this destiny he
had no prospect, and he sought none.

There was over-much drumming, trumpeting, and speech-making, in the
neighbourhood where he dwelt; but he had nothing to do with that.
Such clash and uproar came from the Bigwig family, at the
unaccountable proceedings of which race, he marvelled much.  They
set up the strangest statues, in iron, marble, bronze, and brass,
before his door; and darkened his house with the legs and tails of
uncouth images of horses.  He wondered what it all meant, smiled in
a rough good-humoured way he had, and kept at his hard work.

The Bigwig family (composed of all the stateliest people
thereabouts, and all the noisiest) had undertaken to save him the
trouble of thinking for himself, and to manage him and his affairs.
``Why truly,'' said he, ``I have little time upon my hands; and if you
will be so good as to take care of me, in return for the money I pay
over''---for the Bigwig family were not above his money---``I shall be
relieved and much obliged, considering that you know best.''  Hence
the drumming, trumpeting, and speech-making, and the ugly images of
horses which he was expected to fall down and worship.

``I don't understand all this,'' said he, rubbing his furrowed brow
confusedly.  ``But it \emph{has} a meaning, maybe, if I could find it out.''

``It means,'' returned the Bigwig family, suspecting something of what
he said, ``honour and glory in the highest, to the highest merit.''

``Oh!'' said he.  And he was glad to hear that.

But, when he looked among the images in iron, marble, bronze, and
brass, he failed to find a rather meritorious countryman of his,
once the son of a Warwickshire wool-dealer, or any single countryman
whomsoever of that kind.  He could find none of the men whose
knowledge had rescued him and his children from terrific and
disfiguring disease, whose boldness had raised his forefathers from
the condition of serfs, whose wise fancy had opened a new and high
existence to the humblest, whose skill had filled the working man's
world with accumulated wonders.  Whereas, he did find others whom he
knew no good of, and even others whom he knew much ill of.

``Humph!'' said he.  ``I don't quite understand it.''

So, he went home, and sat down by his fireside to get it out of his
mind.

Now, his fireside was a bare one, all hemmed in by blackened
streets; but it was a precious place to him.  The hands of his wife
were hardened with toil, and she was old before her time; but she
was dear to him.  His children, stunted in their growth, bore traces
of unwholesome nurture; but they had beauty in his sight.  Above all
other things, it was an earnest desire of this man's soul that his
children should be taught.  ``If I am sometimes misled,'' said he,
``for want of knowledge, at least let them know better, and avoid my
mistakes.  If it is hard to me to reap the harvest of pleasure and
instruction that is stored in books, let it be easier to them.''

But, the Bigwig family broke out into violent family quarrels
concerning what it was lawful to teach to this man's children.  Some
of the family insisted on such a thing being primary and
indispensable above all other things; and others of the family
insisted on such another thing being primary and indispensable above
all other things; and the Bigwig family, rent into factions, wrote
pamphlets, held convocations, delivered charges, orations, and all
varieties of discourses; impounded one another in courts Lay and
courts Ecclesiastical; threw dirt, exchanged pummelings, and fell
together by the ears in unintelligible animosity.  Meanwhile, this
man, in his short evening snatches at his fireside, saw the demon
Ignorance arise there, and take his children to itself.  He saw his
daughter perverted into a heavy, slatternly drudge; he saw his son
go moping down the ways of low sensuality, to brutality and crime;
he saw the dawning light of intelligence in the eyes of his babies
so changing into cunning and suspicion, that he could have rather
wished them idiots.

``I don't understand this any the better,'' said he; ``but I think it
cannot be right.  Nay, by the clouded Heaven above me, I protest
against this as my wrong!''

Becoming peaceable again (for his passion was usually short-lived,
and his nature kind), he looked about him on his Sundays and
holidays, and he saw how much monotony and weariness there was, and
thence how drunkenness arose with all its train of ruin.  Then he
appealed to the Bigwig family, and said, ``We are a labouring people,
and I have a glimmering suspicion in me that labouring people of
whatever condition were made---by a higher intelligence than yours,
as I poorly understand it---to be in need of mental refreshment and
recreation.  See what we fall into, when we rest without it.  Come!
Amuse me harmlessly, show me something, give me an escape!''

But, here the Bigwig family fell into a state of uproar absolutely
deafening.  When some few voices were faintly heard, proposing to
show him the wonders of the world, the greatness of creation, the
mighty changes of time, the workings of nature and the beauties of
art---to show him these things, that is to say, at any period of his
life when he could look upon them---there arose among the Bigwigs
such roaring and raving, such pulpiting and petitioning, such
maundering and memorialising, such name-calling and dirt-throwing,
such a shrill wind of parliamentary questioning and feeble replying-%
-where ``I dare not'' waited on ``I would''---that the poor fellow stood
aghast, staring wildly around.

``Have I provoked all this,'' said he, with his hands to his
affrighted ears, ``by what was meant to be an innocent request,
plainly arising out of my familiar experience, and the common
knowledge of all men who choose to open their eyes?  I don't
understand, and I am not understood.  What is to come of such a
state of things!''

He was bending over his work, often asking himself the question,
when the news began to spread that a pestilence had appeared among
the labourers, and was slaying them by thousands.  Going forth to
look about him, he soon found this to be true.  The dying and the
dead were mingled in the close and tainted houses among which his
life was passed.  New poison was distilled into the always murky,
always sickening air.  The robust and the weak, old age and infancy,
the father and the mother, all were stricken down alike.

What means of flight had he?  He remained there, where he was, and
saw those who were dearest to him die.  A kind preacher came to him,
and would have said some prayers to soften his heart in his gloom,
but he replied:

``O what avails it, missionary, to come to me, a man condemned to
residence in this foetid place, where every sense bestowed upon me
for my delight becomes a torment, and where every minute of my
numbered days is new mire added to the heap under which I lie
oppressed!  But, give me my first glimpse of Heaven, through a
little of its light and air; give me pure water; help me to be
clean; lighten this heavy atmosphere and heavy life, in which our
spirits sink, and we become the indifferent and callous creatures
you too often see us; gently and kindly take the bodies of those who
die among us, out of the small room where we grow to be so familiar
with the awful change that even its sanctity is lost to us; and,
Teacher, then I will hear---none know better than you, how willingly-%
-of Him whose thoughts were so much with the poor, and who had
compassion for all human sorrow!''

He was at work again, solitary and sad, when his Master came and
stood near to him dressed in black.  He, also, had suffered heavily.
His young wife, his beautiful and good young wife, was dead; so,
too, his only child.

``Master, 'tis hard to bear---I know it---but be comforted.  I would
give you comfort, if I could.''

The Master thanked him from his heart, but, said he, ``O you
labouring men!  The calamity began among you.  If you had but lived
more healthily and decently, I should not be the widowed and bereft
mourner that I am this day.''

``Master,'' returned the other, shaking his head, ``I have begun to
understand a little that most calamities will come from us, as this
one did, and that none will stop at our poor doors, until we are
united with that great squabbling family yonder, to do the things
that are right.  We cannot live healthily and decently, unless they
who undertook to manage us provide the means.  We cannot be
instructed unless they will teach us; we cannot be rationally
amused, unless they will amuse us; we cannot but have some false
gods of our own, while they set up so many of theirs in all the
public places.  The evil consequences of imperfect instruction, the
evil consequences of pernicious neglect, the evil consequences of
unnatural restraint and the denial of humanising enjoyments, will
all come from us, and none of them will stop with us.  They will
spread far and wide.  They always do; they always have done---just
like the pestilence.  I understand so much, I think, at last.''

But the Master said again, ``O you labouring men!  How seldom do we
ever hear of you, except in connection with some trouble!''

``Master,'' he replied, ``I am Nobody, and little likely to be heard of
(nor yet much wanted to be heard of, perhaps), except when there is
some trouble.  But it never begins with me, and it never can end
with me.  As sure as Death, it comes down to me, and it goes up from
me.''

There was so much reason in what he said, that the Bigwig family,
getting wind of it, and being horribly frightened by the late
desolation, resolved to unite with him to do the things that were
right---at all events, so far as the said things were associated with
the direct prevention, humanly speaking, of another pestilence.
But, as their fear wore off, which it soon began to do, they resumed
their falling out among themselves, and did nothing.  Consequently
the scourge appeared again---low down as before---and spread
avengingly upward as before, and carried off vast numbers of the
brawlers.  But not a man among them ever admitted, if in the least
degree he ever perceived, that he had anything to do with it.

So Nobody lived and died in the old, old, old way; and this, in the
main, is the whole of Nobody's story.

Had he no name, you ask?  Perhaps it was Legion.  It matters little
what his name was.  Let us call him Legion.

If you were ever in the Belgian villages near the field of Waterloo,
you will have seen, in some quiet little church, a monument erected
by faithful companions in arms to the memory of Colonel A, Major B,
Captains C, D and E, Lieutenants F and G, Ensigns H, I and J, seven
non-commissioned officers, and one hundred and thirty rank and file,
who fell in the discharge of their duty on the memorable day.  The
story of Nobody is the story of the rank and file of the earth.
They bear their share of the battle; they have their part in the
victory; they fall; they leave no name but in the mass.  The march
of the proudest of us, leads to the dusty way by which they go.  O!
Let us think of them this year at the Christmas fire, and not forget
them when it is burnt out.


\end{document}


% End of The Project Gutenberg Etext of Some Christmas Stories by Dickens
% 
